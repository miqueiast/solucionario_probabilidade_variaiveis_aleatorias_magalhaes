\documentclass[12pt]{article}
\usepackage[utf8]{inputenc}
\usepackage{amsmath}
\usepackage{amssymb}
\usepackage{amsfonts}
\usepackage{amsthm}
\usepackage{geometry}
\geometry{a4paper, margin=1in}

\title{Exercícios do Livro Magalhães - Capítulo 03\\\large{Seção 3.4}}
\author{Aluno: Miqueias T}
\date{27 de Agosto de 2025}

\begin{document}

\maketitle

\section*{Introdução}
Este documento apresenta a resolução de exercícios teóricos selecionados sobre a definição formal de variáveis aleatórias, independência de eventos e propriedades de funções de distribuição. Os problemas exploram conceitos fundamentais da teoria da medida e da probabilidade.

\section{Exercício 1}

\begin{quote}
Sendo X e Y variáveis aleatórias em $(\Omega, \mathcal{F}, P)$, mostre que $\min(X,Y)$ e $\max(X,Y)$ também são variáveis aleatórias.
\end{quote}

\subsection*{Resolução}
Para mostrar que uma função $Z: \Omega \to \mathbb{R}$ é uma variável aleatória, precisamos provar que o conjunto $\{\omega \in \Omega : Z(\omega) \le z\}$ pertence à $\sigma$-álgebra $\mathcal{F}$ para todo $z \in \mathbb{R}$.

\paragraph{Prova para $Z = \max(X,Y)$:}
Vamos analisar o evento $\{Z \le z\}$, ou seja, $\{\max(X,Y) \le z\}$.
O máximo de dois números é menor ou igual a $z$ se, e somente se, ambos os números são menores ou iguais a $z$. Portanto, podemos reescrever o evento da seguinte forma:
\[ \{\max(X,Y) \le z\} = \{X \le z \text{ e } Y \le z\} \]
Este evento corresponde à interseção de dois outros eventos:
\[ \{\omega \in \Omega : X(\omega) \le z\} \cap \{\omega \in \Omega : Y(\omega) \le z\} \]
Por definição, como X e Y são variáveis aleatórias, os conjuntos $\{X \le z\}$ e $\{Y \le z\}$ pertencem a $\mathcal{F}$ para qualquer $z \in \mathbb{R}$.
Uma das propriedades fundamentais de uma $\sigma$-álgebra é que ela é fechada sob interseções finitas. Como ambos os conjuntos estão em $\mathcal{F}$, a sua interseção também deve estar em $\mathcal{F}$.
Logo, $\{\max(X,Y) \le z\} \in \mathcal{F}$ para todo $z \in \mathbb{R}$, o que prova que $\max(X,Y)$ é uma variável aleatória.

\paragraph{Prova para $W = \min(X,Y)$:}
Agora, vamos analisar o evento $\{W \le z\}$, ou seja, $\{\min(X,Y) \le z\}$.
O mínimo de dois números é menor ou igual a $z$ se, e somente se, pelo menos um dos números é menor ou igual a $z$. Assim, podemos reescrever o evento como uma união:
\[ \{\min(X,Y) \le z\} = \{X \le z \text{ ou } Y \le z\} \]
Este evento corresponde à união de dois outros eventos:
\[ \{\omega \in \Omega : X(\omega) \le z\} \cup \{\omega \in \Omega : Y(\omega) \le z\} \]
Novamente, como X e Y são variáveis aleatórias, os conjuntos $\{X \le z\}$ e $\{Y \le z\}$ pertencem a $\mathcal{F}$.
Uma $\sigma$-álgebra também é fechada sob uniões finitas. Portanto, a união dos dois conjuntos também pertence a $\mathcal{F}$.
Logo, $\{\min(X,Y) \le z\} \in \mathcal{F}$ para todo $z \in \mathbb{R}$, o que prova que $\min(X,Y)$ é uma variável aleatória.
\hfill \qedsymbol

\pagebreak

\section{Exercício 3}

\begin{quote}
Mostre que $A_1, A_2, \dots, A_n$ são eventos independentes se e só se $I_{A_1}, I_{A_2}, \dots, I_{A_n}$ forem variáveis aleatórias independentes.
\end{quote}

\subsection*{Resolução}
Seja $I_{A_i}$ a variável aleatória indicadora do evento $A_i$, tal que $I_{A_i}=1$ se $A_i$ ocorre, e $I_{A_i}=0$ caso contrário. Note que $P(I_{A_i}=1) = P(A_i)$ e $P(I_{A_i}=0) = 1 - P(A_i) = P(A_i^c)$.

\paragraph{($\Rightarrow$) Se os eventos são independentes, as v.a. indicadoras são independentes.}
Assumimos que $A_1, \dots, A_n$ são eventos independentes. Para mostrar que as variáveis aleatórias $I_{A_1}, \dots, I_{A_n}$ são independentes, devemos mostrar que para qualquer escolha de $x_1, \dots, x_n \in \{0,1\}$:
\[ P(I_{A_1}=x_1, \dots, I_{A_n}=x_n) = \prod_{i=1}^{n} P(I_{A_i}=x_i) \]
O evento $\{I_{A_1}=x_1, \dots, I_{A_n}=x_n\}$ corresponde à interseção de $n$ eventos, onde o $i$-ésimo evento é $A_i$ se $x_i=1$, ou $A_i^c$ se $x_i=0$. Vamos chamar esses eventos de $B_i$.
\[ \{I_{A_1}=x_1, \dots, I_{A_n}=x_n\} = \bigcap_{i=1}^n B_i, \quad \text{onde } B_i = \begin{cases} A_i & \text{se } x_i=1 \\ A_i^c & \text{se } x_i=0 \end{cases} \]
Como os eventos $A_1, \dots, A_n$ são independentes, qualquer coleção formada por $A_i$ ou seus complementos $A_i^c$ também será de eventos independentes. Portanto, os eventos $B_1, \dots, B_n$ são independentes.
Assim,
\begin{align*}
    P(I_{A_1}=x_1, \dots, I_{A_n}=x_n) &= P\left(\bigcap_{i=1}^n B_i\right) \\
    &= \prod_{i=1}^n P(B_i) \quad (\text{pela independência dos } B_i) \\
    &= \prod_{i=1}^n P(I_{A_i}=x_i)
\end{align*}
Isso prova que as variáveis aleatórias indicadoras são independentes.

\paragraph{($\Leftarrow$) Se as v.a. indicadoras são independentes, os eventos são independentes.}
Assumimos que $I_{A_1}, \dots, I_{A_n}$ são variáveis aleatórias independentes. Para mostrar que os eventos $A_1, \dots, A_n$ são independentes, devemos mostrar que para qualquer subconjunto de índices $J \subseteq \{1, \dots, n\}$:
\[ P\left(\bigcap_{j \in J} A_j\right) = \prod_{j \in J} P(A_j) \]
O evento $\bigcap_{j \in J} A_j$ é precisamente o evento em que $I_{A_j}=1$ para todo $j \in J$.
\[ P\left(\bigcap_{j \in J} A_j\right) = P(I_{A_j}=1 \text{ para todo } j \in J) \]
Como as variáveis indicadoras são independentes, a probabilidade da ocorrência conjunta é o produto das probabilidades marginais:
\begin{align*}
    P(I_{A_j}=1 \text{ para todo } j \in J) &= \prod_{j \in J} P(I_{A_j}=1) \\
    &= \prod_{j \in J} P(A_j)
\end{align*}
Isso prova que os eventos são independentes. \hfill \qedsymbol

\pagebreak

\section{Exercício 5}

\begin{quote}
Seja X uma variável aleatória com função de distribuição $F_X$. Seja $Y=h(X)$ com função de distribuição $F_Y$. Mostre que:
\begin{itemize}
    \item[a.] $F_{X,Y}(x,y) = \min\{F_X(x), F_Y(y)\}$, se h é monótona crescente.
    \item[b.] $F_{X,Y}(x,y) = \max\{F_X(x) + F_Y(y) - 1, 0\}$, se h é monótona decrescente.
\end{itemize}
\end{quote}

\subsection*{Resolução}
A função de distribuição conjunta é $F_{X,Y}(x,y) = P(X \le x, Y \le y)$. Substituindo $Y=h(X)$, temos $F_{X,Y}(x,y) = P(X \le x, h(X) \le y)$.

\subsubsection*{a. h é monótona crescente}
Se $h$ é monótona crescente, ela possui uma inversa $h^{-1}$ que também é monótona crescente. A desigualdade $h(X) \le y$ é equivalente a $X \le h^{-1}(y)$.
\begin{align*}
    F_{X,Y}(x,y) &= P(X \le x \text{ e } X \le h^{-1}(y)) \\
    &= P(X \le \min\{x, h^{-1}(y)\}) \\
    &= F_X(\min\{x, h^{-1}(y)\})
\end{align*}
Como a função $F_X$ é não-decrescente, $F_X(\min\{a,b\}) = \min\{F_X(a), F_X(b)\}$. Assim:
\[ F_{X,Y}(x,y) = \min\{F_X(x), F_X(h^{-1}(y))\} \]
Agora, vamos analisar o termo $F_Y(y)$.
\[ F_Y(y) = P(Y \le y) = P(h(X) \le y) = P(X \le h^{-1}(y)) = F_X(h^{-1}(y)) \]
Substituindo isso na expressão anterior, obtemos o resultado desejado:
\[ F_{X,Y}(x,y) = \min\{F_X(x), F_Y(y)\}. \]

\subsubsection*{b. h é monótona decrescente}
Se $h$ é monótona decrescente, sua inversa $h^{-1}$ também é. A desigualdade $h(X) \le y$ é equivalente a $X \ge h^{-1}(y)$ (a ordem da desigualdade inverte).
\begin{align*}
    F_{X,Y}(x,y) &= P(X \le x \text{ e } X \ge h^{-1}(y)) \\
    &= P(h^{-1}(y) \le X \le x)
\end{align*}
Se $x < h^{-1}(y)$, o intervalo é vazio e a probabilidade é 0. Caso contrário, para uma v.a. contínua, a probabilidade é $F_X(x) - F_X(h^{-1}(y))$.
Vamos analisar a expressão dada: $\max\{F_X(x) + F_Y(y) - 1, 0\}$.
Primeiro, calculamos $F_Y(y)$:
\[ F_Y(y) = P(Y \le y) = P(h(X) \le y) = P(X \ge h^{-1}(y)) = 1 - P(X < h^{-1}(y)) \]
Assumindo que X é contínua, $P(X < h^{-1}(y)) = F_X(h^{-1}(y))$. Portanto, $F_Y(y) = 1 - F_X(h^{-1}(y))$.
Agora, substituímos na expressão:
\begin{align*}
    F_X(x) + F_Y(y) - 1 &= F_X(x) + (1 - F_X(h^{-1}(y))) - 1 \\
    &= F_X(x) - F_X(h^{-1}(y))
\end{align*}
Isso corresponde a $P(h^{-1}(y) \le X \le x)$. O termo $\max\{\dots, 0\}$ garante que a probabilidade seja não-negativa, cobrindo o caso em que o intervalo é vazio.
Portanto, a identidade está correta. \hfill \qedsymbol

\pagebreak

\section{Exercício 7}
\begin{quote}
Mostre que, para variáveis contínuas X e Y, temos
\[ F_X(x) + F_Y(y) - 1 \le F_{X,Y}(x,y) \le \sqrt{F_X(x)F_Y(y)}; \forall x,y \in \mathbb{R}. \]
\end{quote}

\subsection*{Resolução}
Vamos provar as duas desigualdades separadamente. Sejam os eventos $A = \{X \le x\}$ e $B = \{Y \le y\}$. Pelas definições das funções de distribuição, temos:
\begin{itemize}
    \item $P(A) = F_X(x)$
    \item $P(B) = F_Y(y)$
    \item $P(A \cap B) = F_{X,Y}(x,y)$
\end{itemize}

\paragraph{Prova da Desigualdade à Esquerda: $F_X(x) + F_Y(y) - 1 \le F_{X,Y}(x,y)$}
Esta desigualdade é uma consequência direta da fórmula de união de probabilidades (ou desigualdade de Bonferroni).
A probabilidade da união de dois eventos é:
\[ P(A \cup B) = P(A) + P(B) - P(A \cap B) \]
Sabemos que qualquer probabilidade deve ser menor ou igual a 1, então $P(A \cup B) \le 1$.
\[ P(A) + P(B) - P(A \cap B) \le 1 \]
Reorganizando os termos para isolar $P(A \cap B)$:
\[ P(A) + P(B) - 1 \le P(A \cap B) \]
Substituindo pelas funções de distribuição, obtemos a desigualdade desejada:
\[ F_X(x) + F_Y(y) - 1 \le F_{X,Y}(x,y). \]

\paragraph{Prova da Desigualdade à Direita: $F_{X,Y}(x,y) \le \sqrt{F_X(x)F_Y(y)}$}
Esta desigualdade pode ser provada usando a desigualdade de Cauchy-Schwarz para variáveis aleatórias.
Sejam $I_A$ e $I_B$ as variáveis aleatórias indicadoras dos eventos A e B. A desigualdade de Cauchy-Schwarz para esperanças afirma que $(E[UV])^2 \le E[U^2]E[V^2]$.
Aplicando para $U=I_A$ e $V=I_B$:
\[ (E[I_A I_B])^2 \le E[I_A^2] E[I_B^2] \]
Uma propriedade das variáveis indicadoras é que $I_A^2=I_A$ (pois $0^2=0$ e $1^2=1$). Da mesma forma, $I_B^2=I_B$. Além disso, o produto $I_A I_B$ é a indicadora da interseção, $I_{A \cap B}$. A esperança de uma variável indicadora é a probabilidade do evento correspondente.
\begin{itemize}
    \item $E[I_A I_B] = E[I_{A \cap B}] = P(A \cap B)$
    \item $E[I_A^2] = E[I_A] = P(A)$
    \item $E[I_B^2] = E[I_B] = P(B)$
\end{itemize}
Substituindo na desigualdade de Cauchy-Schwarz:
\[ (P(A \cap B))^2 \le P(A) P(B) \]
Tirando a raiz quadrada de ambos os lados (como probabilidades são não-negativas, a desigualdade se mantém):
\[ P(A \cap B) \le \sqrt{P(A)P(B)} \]
Substituindo de volta pelas funções de distribuição:
\[ F_{X,Y}(x,y) \le \sqrt{F_X(x)F_Y(y)}. \]
Isso completa a demonstração das duas desigualdades. \hfill \qedsymbol

\end{document}