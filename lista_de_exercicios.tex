% Options for packages loaded elsewhere
\PassOptionsToPackage{unicode}{hyperref}
\PassOptionsToPackage{hyphens}{url}
\documentclass[
  10pt,
]{article}
\usepackage{xcolor}
\usepackage[left=2cm, right=2cm, top=2.5cm, bottom=2.5cm]{geometry}
\usepackage{amsmath,amssymb}
\setcounter{secnumdepth}{-\maxdimen} % remove section numbering
\usepackage{iftex}
\ifPDFTeX
  \usepackage[T1]{fontenc}
  \usepackage[utf8]{inputenc}
  \usepackage{textcomp} % provide euro and other symbols
\else % if luatex or xetex
  \usepackage{unicode-math} % this also loads fontspec
  \defaultfontfeatures{Scale=MatchLowercase}
  \defaultfontfeatures[\rmfamily]{Ligatures=TeX,Scale=1}
\fi
\usepackage{lmodern}
\ifPDFTeX\else
  % xetex/luatex font selection
\fi
% Use upquote if available, for straight quotes in verbatim environments
\IfFileExists{upquote.sty}{\usepackage{upquote}}{}
\IfFileExists{microtype.sty}{% use microtype if available
  \usepackage[]{microtype}
  \UseMicrotypeSet[protrusion]{basicmath} % disable protrusion for tt fonts
}{}
\makeatletter
\@ifundefined{KOMAClassName}{% if non-KOMA class
  \IfFileExists{parskip.sty}{%
    \usepackage{parskip}
  }{% else
    \setlength{\parindent}{0pt}
    \setlength{\parskip}{6pt plus 2pt minus 1pt}}
}{% if KOMA class
  \KOMAoptions{parskip=half}}
\makeatother
\usepackage{graphicx}
\makeatletter
\newsavebox\pandoc@box
\newcommand*\pandocbounded[1]{% scales image to fit in text height/width
  \sbox\pandoc@box{#1}%
  \Gscale@div\@tempa{\textheight}{\dimexpr\ht\pandoc@box+\dp\pandoc@box\relax}%
  \Gscale@div\@tempb{\linewidth}{\wd\pandoc@box}%
  \ifdim\@tempb\p@<\@tempa\p@\let\@tempa\@tempb\fi% select the smaller of both
  \ifdim\@tempa\p@<\p@\scalebox{\@tempa}{\usebox\pandoc@box}%
  \else\usebox{\pandoc@box}%
  \fi%
}
% Set default figure placement to htbp
\def\fps@figure{htbp}
\makeatother
\setlength{\emergencystretch}{3em} % prevent overfull lines
\providecommand{\tightlist}{%
  \setlength{\itemsep}{0pt}\setlength{\parskip}{0pt}}
\usepackage{fancyhdr}
\pagestyle{fancy}
\fancyhead[L]{Miqueias}
\fancyhead[C]{Lista de Exercícios}
\fancyhead[R]{\today}
\fancyfoot[C]{\thepage}
\usepackage{bookmark}
\IfFileExists{xurl.sty}{\usepackage{xurl}}{} % add URL line breaks if available
\urlstyle{same}
\hypersetup{
  hidelinks,
  pdfcreator={LaTeX via pandoc}}

\author{}
\date{\vspace{-2.5em}}

\begin{document}

\section{Probabilidade: Um Curso
Introdutório}\label{probabilidade-um-curso-introdutuxf3rio}

\subsection{4.5 EXERCÍCIOS}\label{exercuxedcios}

\begin{enumerate}
\def\labelenumi{\arabic{enumi}.}
\item
  Sabe-se que os parafusos produzidos por uma certa companhia são
  defeituosos com probabilidade 0,01, independentemente uns dos outros
  (isto é, a fração não-conforme de parafusos na produção é 0,01). A
  companhia vende os parafusos em pacotes de dez unidades e oferece uma
  garantia de devolução do dinheiro caso existam dois ou mais parafusos
  defeituosos no pacote com dez parafusos. (a) Qual a proporção de
  pacotes vendidos para os quais a companhia deve efetuar devolução do
  dinheiro? (b) Supondo que o número de parafusos defeituosos num
  determinado pacote é independente dos demais pacotes, qual a
  probabilidade de que uma pessoa que compra dez pacotes de parafusos
  tenha que retornar à companhia para devolução do dinheiro?
\item
  Suponha que o número de erros tipográficos em uma única página de um
  livro tem distribuição de Poisson com parâmetro
  \(\lambda = \frac{1}{2}\). (a) Calcule a probabilidade de existir
  exatamente dois erros tipográficos em uma página. (b) Calcule a
  probabilidade de que exista pelo menos um erro em uma página. (c)
  Suponha agora que o livro em questão possui 200 páginas. Qual a
  probabilidade de não existir erros tipográficos neste livro?
\item
  Uma moeda, cuja probabilidade de cara em um lançamento é
  \(p, 0 < p < 1\), é lançada indefinidamente até que ocorra uma
  seqüência de cinco caras (seguidas). Um estudante resolve modelar esse
  experimento definindo a variável aleatória \(X\) igual ao número de
  lançamentos da moeda até que ocorra a seqüência de cinco caras. Ele
  então reclama do seguinte modo: para \(k = 5\), \(P(X = k) = p^k\),
  pois devemos ter cara nos cinco primeiros lançamentos. Para \(k = 6\),
  \(P(X = k) = (1 - p)^p\), pois devemos ter coroa no primeiro
  lançamento e cara nos cinco lançamentos seguintes. Assim, concluiu
  que\\
  \[P(X = k) =  
  \begin{cases}  
  0 & , k < 5, \\  
  (1 - p)^{k-5}p^k & , k \geq 5,  
  \end{cases}\]\\
  e, portanto, \(X^* = X - 5\) tem distribuição geométrica de parâmetro
  \(p^k\). Discuta a proposta do estudante abordando as características
  do modelo proposto e a validade da expressão \(P(X = k)\).
\item
  Suponha que uma particular característica de uma pessoa (por exemplo,
  cor dos olhos) é determinada por um par de genes. Suponha ainda que
  \(D\) representa o gene dominante e \(R\) o gene recessivo, de tal
  forma que uma pessoa com genes \(DD\) é dominante pura, uma com \(RR\)
  é recessiva pura e uma com \(RD\) ou \(DR\) é híbrida, e que uma
  pessoa dominante pura e uma pessoa híbrida têm aparências semelhantes
  quanto a esta característica. Se ambos os membros de um casal são
  híbridos quanto a uma particular característica e se o casal possui
  quatro filhos, qual a probabilidade de que três das quatro crianças
  tenham fenótipo determinado pelo gene dominante? (Lembre-se que cada
  criança recebe um gene de cada um dos pais.)
\item
  Suponha que a probabilidade de que um item produzido por uma máquina
  seja defeituoso é 0,1. Determine a probabilidade de que uma amostra de
  dez itens conterá no máximo um item defeituoso. Compare os resultados
  obtidos pelas distribuições binomial e Poisson.
\item
  Com a finalidade de aumentar a arrecadação do seu sistema de loterias,
  a Caixa Econômica Federal implantou um novo jogo denominado Loto II,
  no qual o apostador escolhe seis dezenas do conjunto
  \((01, 02, \ldots, 50)\). Toda semana, a Caixa sorteia seis dezenas
  desse mesmo conjunto e atribui prémios aos acertadores da: (a) Sena --
  as seis dezenas sorteadas, (b) Quina -- cinco das dezenas sorteadas,
  (c) Quadra -- quatro das dezenas sorteadas. Determine a probabilidade
  de que uma pessoa que aposta na Loto II ganhe algum dos prémios
  oferecidos.
\item
  Um apostador da Loto II (descrita no exercício 6) é acusado de
  conhecer previamente o resultado das dezenas sorteadas, uma vez que
  ganhou doze prémios da Loto II em um único ano, fato considerado
  extremamente raro. Construa um modelo probabilístico adequado ao
  problema e verifique a validade de tal acusação.
\item
  Considere um experimento que consiste em contar o número de partículas
  alfa emitidas, num intervalo de tempo de um segundo, por um grama de
  material radioativo. Sabe-se por experiência passada que, em média,
  3,2 de tais partículas são emitidas por segundo. Determine a
  probabilidade (aproximada) de que não mais que duas partículas alfa
  sejam emitidas.
\item
  Um comprador adquire componentes eletrônicos em lotes de tamanho dez.
  Este comprador adota a seguinte postura: inspeciona três componentes
  aleatoriamente de um lote e o aceita apenas se todos os três
  componentes não forem defeituosos. Se 30\% dos lotes têm quatro
  componentes defeituosos e 70\% têm apenas um, qual a proporção de
  lotes que o comprador rejeita?
\item
  Dois amigos, Paulo e Pedro, resolvem ir juntos a uma casa lotérica
  para apostar na Loto II. Paulo apostou nas dez últimas semanas, ao
  passo que Pedro nunca apostou na loteria em questão. Paulo argumenta
  que Pedro deve apostar nas mesmas dezenas que ele, pois julga que o
  fato de já ter participado do jogo algumas vezes e nunca ter ganho
  prêmio algum torna mais eminente sua vitória no próximo sorteio,
  relativamente a Pedro. Em outras palavras, Paulo argumenta que o fato
  de já ter apostado dez vezes na Loto II e nunca ter ganho prêmio algum
  aumenta a probabilidade de obter algum prêmio no próximo sorteio.
  Discuta a afirmação feita por Paulo propondo um modelo probabilístico
  para a situação descrita.
\item
  Considere um júri constituído por dez jurados que adota a seguinte
  regra para determinar o veredito: se pelo menos oito dos jurados
  julgarem o réu culpado, este será condenado, caso contrário, o réu não
  será condenado. Suponha que cada jurado dá seu veredito
  independentemente dos demais e que cada um toma a decisão correta com
  probabilidade \(\theta\). Suponha ainda que o representa a
  probabilidade do réu ser culpado. Determine a probabilidade do júri
  tomar uma decisão correta.
\item
  Um jornaleiro compra jornais a dez centavos e os vende a quinze
  centavos. Contudo, ele não pode devolver os jornais não vendidos. Se a
  demanda diária de jornais é uma variável aleatória binomial com
  \(n = 10\) e \(p = \frac{1}{3}\), quantos jornais, aproximadamente,
  ele deve comprar de maneira a maximizar seu lucro esperado?
\item
  Suponha que um motor de um avião falha, em vôo, com probabilidade
  \(1 - p\), independentemente dos demais motores. Se um avião precisa
  de que a maioria dos seus motores funcione para realizar um vôo
  bem-sucedido, para que valores de \(p\) um avião com cinco motores é
  preferível a um avião com três motores?
\item
  Um sistema de satélites consiste de \(N\) componentes e funciona em
  qualquer dia se pelo menos \(k\) dos \(N\) componentes funcionarem
  nesse dia. Em dias chuvosos, cada componente funciona
  independentemente com probabilidade \(p_1\), ao passo que em dias não
  chuvosos os componentes funcionam independentemente com probabilidade
  \(p_2\). Se a probabilidade de chover amanhã é \(\alpha\), qual é a
  probabilidade de que o sistema de satélites irá funcionar amanhã?
\item
  Mostre que \(\sum_{k=0}^{n} \binom{n}{k}^2 = \binom{2n}{n}\).
\end{enumerate}

Dica: defina uma variável aleatória com distribuição hipergeométrica de
parâmetros \(2n, n, n\).

\begin{enumerate}
\def\labelenumi{\arabic{enumi}.}
\setcounter{enumi}{15}
\item
  Para uma variável aleatória \(N\) assumindo valores inteiros
  não-negativos, mostre que
  \(E(N) = \sum_{i=1}^{\infty} P_i(N \geq i)\).
\item
  Seja \(N\) uma variável aleatória que assume valores inteiros
  não-negativos. Mostre que
  \(\sum_{i=0}^{\infty} iP_i(N > i) = \frac{1}{2}(E[N^2] - E[N])\).
\item
  Seja \(X\) uma variável aleatória com valor esperado \(\mu\) e
  variância \(\sigma^2\). Seja \(Y = \frac{X - \mu}{\sigma^2}\).
  Determine: (a) \(E[Y]\). (b) \(\text{Var}[Y]\).
\end{enumerate}

A variável \(Y\) assim obtida é denominada variável padronizada (neste
caso, \(Y\) é obtida padronizando-se \(X\)) e é de grande importância em
várias técnicas de análise estatística.

\begin{enumerate}
\def\labelenumi{\arabic{enumi}.}
\setcounter{enumi}{18}
\item
  Mostre que para uma variável aleatória \(X\) com distribuição
  hipergeométrica de parâmetros \(N, m, n\), a variância é dada pela
  fórmula (4.8).
\item
  Mostre que:
  \[ \binom{M}{k} \binom{N - M}{n - k} / \binom{N}{n} \to \binom{n}{k} p^k (1 - p)^{n - k}, \]
  quando \(N, M \to \infty\) tais que \(\frac{M}{N} \to p\). Interprete
  o resultado.
\item
  Seja \(X\) uma variável aleatória Poisson com parâmetro \(\lambda\).
  Mostre que \(P(X = k)\) atinge o valor máximo para \(k = [\lambda]\),
  onde \([\lambda]\) denota o maior inteiro menor ou igual a
  \(\lambda\).
\item
  Seja \(X\) uma variável aleatória binomial com parâmetros \(n e p\).
  Mostre que \(P(X = k)\) atinge o valor máximo para \(k = [(n + 1)p]\).
\item
  Seja \(X\) uma variável aleatória com distribuição de Poisson com
  parâmetro \(\lambda\). Determine \(P(X \in A)\), onde
  \(A = \{0, 2, 4, 6, 8, \ldots\}\).
\item
  Considere \(X\) uma variável aleatória Poisson com parâmetro
  \(\lambda\). (a) Mostre que \(E[X^n] = \lambda E[(X + 1)^{n-1}]\). (b)
  Calcule \(E[X^4]\). (c) Determine \(E[X]\), para \(0 < \lambda < 1\).
  (d) Determine \(E[\cos \pi X]\) e \(\text{Var}[\cos \pi X]\).
\item
  Suponha que o número de eventos que ocorrem em um intervalo de tempo
  especificado é uma variável aleatória de Poisson com parâmetro
  \(\lambda\). Se cada evento é contado com probabilidade \(p\),
  independentemente de qualquer outro evento, mostre que o número de
  eventos que são contados é uma variável aleatória Poisson com
  parâmetro \(\lambda p\).
\item
  Para uma variável aleatória com distribuição hipergeométrica com
  parâmetros \(N, m e n\), determine: \[ \frac{P(X = k+1)}{P(X = k)} \].
\item
  Se \(X \circ Y\) são variáveis aleatórias de Poisson independentes com
  parâmetros \(\lambda_1 \circ \lambda_2\), respectivamente, calcule a
  distribuição de \(X + Y\).
\item
  Sejam \(X \circ Y\) variáveis aleatórias binomiais independentes com
  parâmetros \((n, p) \circ (m, p)\), respectivamente. Determine a
  distribuição de \(X + Y\).
\item
  Considere novamente as variáveis do exercício 27. Calcule a
  distribuição condicional de \(X\) dado que \(X + Y = n\).
\item
  Considere novamente as variáveis \(X \circ Y\) do exercício 28, com a
  restrição \(n = m\), isto é, \(X \circ Y\) têm a mesma distribuição
  binomial \((n, p)\). Determine a distribuição condicional de \(X\)
  dado que \(X + Y = m\).
\item
  Suponha que \(X \circ Y\) são variáveis aleatórias geométricas
  independentes com mesmo parâmetro \(p\). Calcule
  \[ P\{X = i|X + Y = n\} \].
\item
  Suponha que ensaios independentes, cada um tendo probabilidade \(p\)
  de sucesso, \(0 < p < 1\), são realizadas até que um total de \(R\)
  sucessos seja acumulado. Seja \(X\) o número de ensaios necessários
  para se obter o total de \(R\) sucessos. Determine a distribuição de
  \(X\) (tikemos, neste caso, que \(X\) possui distribuição binomial
  negativa com parâmetros \(R e p\).
\item
  Considere a variável aleatória \(X\) do exercício anterior. Determine
  \(E(X) \circ \text{Var}(X)\).
\end{enumerate}

Dica: Observe que \(X = Y_1 + Y_2 + \cdots + Y_R\), onde \(Y_i\) tem
distribuição geométrica de parâmetro \(p\), \(1 \leq i \leq R\).

\begin{enumerate}
\def\labelenumi{\arabic{enumi}.}
\setcounter{enumi}{33}
\item
  Se ensaios independentes, cada um deles resultando em sucesso com
  probabilidade \(p\), são realizados indefinidamente, qual a
  probabilidade de que \(R\) sucessos ocorram antes de \(M\) fracassos?
\item
  Suponha que existam \(N\) diferentes figurinhas e que um colecionador
  compre sucessivamente envelopes fechados que contém apenas uma delas.
  Admita que a probabilidade de um envelope conter uma dada figurinha é
  \(1/N\) independentemente uns dos outros. Seja \(X\) o número de
  envelopes que o colecionador deve comprar até conseguir completar sua
  coleção. (a) \(X\) tem distribuição binomial negativa? (b) Determine
  \(E(X)\).
\item
  Sejam \(X, Y \in Z\) variáveis aleatórias independentes com a mesma
  distribuição geométrica de parâmetro \(p\). Determine:\\
\end{enumerate}

\begin{enumerate}
\def\labelenumi{(\alph{enumi})}
\tightlist
\item
  \(P(X = Y)\). (b) \(P(X \geq 2Y)\). (c) \(P(X + Y \leq Z)\).
\end{enumerate}

\begin{enumerate}
\def\labelenumi{\arabic{enumi}.}
\setcounter{enumi}{36}
\item
  Sejam \(X\) uma variável aleatória binomial negativa com parâmetros
  \(R e p\), e \(Y\) uma variável aleatória binomial com parâmetros
  \(n e p\). Mostre que \(P(X > n) = P(Y < r)\).
\item
  Mostre que, se a variável aleatória \(X\) tem distribuição de Poisson
  com parâmetro \(\lambda\), sua variância é dada pela expressão (4.17).
\item
  Seja \(X\) uma variável aleatória binomial com parâmetros \(n e p\).
  Qual o valor de \(p\) que maximiza \(P(X = k)\),
  \(k = 0, 1, 2, \ldots, n\)? Este é um exemplo de um método estatístico
  usado para estimar \(p\) quando a observação de uma variável aleatória
  binomial (\(n, p\)) é igual a \(k\). Se o valor de \(n\) é conhecido e
  o de \(p\) desconhecido, então estimamos \(p\) escolhendo o valor de
  \(p\) que maximiza \(P(X = k)\). Este método é conhecido como
  estimação de máxima verossimilhança.
\item
  Seja \(X\) uma variável aleatória Poisson com parâmetro \(\lambda\).
  Qual o valor de \(\lambda\) que maximiza \(P(X = k)\), \(k \geq 0\)?
  Em outras palavras, determine o estimador de máxima verossimilhança
  para \(\lambda\).
\item
  Um número desconhecido, digamos \(N\), de animais de uma certa espécie
  habitam uma dada região. Para obter alguma informação sobre o tamanho
  dessa população animal, ecologistas normalmente realizam o seguinte
  experimento: inicialmente capturam um certo número de animais, digamos
  \(m\). Em seguida, os animais capturados são marcados de alguma
  maneira e então devolvidos (soltos) na região em questão. Após um
  certo período de tempo, necessário para que os animais marcados se
  dispersem por toda a região, uma nova captura de \(n\) animais é
  feita. Seja \(X\) o número de animais marcados que são apreendidos na
  segunda captura. Se admitirmos que a população de animais na região se
  manteve inalterada (fixa) entre os dois instantes de captura e que em
  cada um destes instantes os \(N\) animais tinham a mesma probabilidade
  de serem capturados, determine:\\
\end{enumerate}

\begin{enumerate}
\def\labelenumi{(\alph{enumi})}
\tightlist
\item
  \(P(X = i)\) em função de \(N\) (desconhecido).\\
\item
  O estimador de máxima verossimilhança para \(N\) se não capturados i
  animais marcados, isto é, se \(X = i\).
\end{enumerate}

\section{Probabilidade: Um Curso
Moderno}\label{probabilidade-um-curso-moderno}

\subsection{5.4 EXERCÍCIOS}\label{exercuxedcios-1}

\begin{enumerate}
\def\labelenumi{\arabic{enumi}.}
\item
  Se \(X\) é uniformemente distribuída no intervalo \((0, 20)\), calcule
  a probabilidade de: (a) \(X < 3\). (b) \(X > 12\). (c) \(4 < X < 11\).
  (d) \(|X - 3| < 4\).
\item
  Se \(X\) é uma variável aleatória normal com parâmetros \(\mu = 3\) e
  \(\sigma^2 = 9\), determine: (a) \(P\{2 < X < 5\}\). (b)
  \(P\{X > 0\}\). (c) \(P\{|X - 3| > 6\}\).
\item
  Se \(X\) é uma variável aleatória uniformemente distribuída no
  intervalo \((-3, 7)\), determine: (a) A função de distribuição de
  \(X\). (b) \(P\{|X - 1| \leq 2\}\). (c) \(P\{|X| > 3\}\).
\item
  Ônibus chegam a um determinado ponto de parada em intervalos de tempo
  de quinze minutos a partir de 7 horas da manhã, isto é, os ônibus
  chegam ao ponto às 7h00, 7h15, 7h30, 7h45, e assim por diante. Se o
  instante de chegada de um passageiro ao ponto é uniformemente
  distribuído entre 7h00 e 7h30, determine a probabilidade: (a) De que
  ele espere menos que 5 minutos até a chegada de um ônibus. (b) De que
  ele espere mais de 10 minutos até a chegada de um ônibus.
\item
  Seja \(X\) o número de caras observadas em 40 lançamentos de uma moeda
  honesta. Determine a probabilidade de \(X = 20\). Utilize a
  aproximação normal e então compare com o resultado obtido através da
  distribuição binomial.
\item
  Suponha que a duração de uma ligação telefônica em uma cabine pública
  (em minutos) é uma variável aleatória exponencial com parâmetro
  \(\lambda = \frac{1}{10}\). Se uma pessoa chega imediatamente a sua
  frente na cabine telefônica pública, ache a probabilidade de que você
  terá que esperar: (a) mais que 10 minutos; (b) entre 10 e 20 minutos.
\item
  Para determinar a eficiência de uma certa dieta na redução da
  quantidade de colesterol na corrente sanguínea, 100 pessoas são
  submetidas a esta dieta por um intervalo de tempo bastante prolongado.
  Em seguida, são registrados os níveis de colesterol destas pessoas. O
  nutricionista responsável pelo experimento decidiu endossar a dieta se
  pelo menos 65\% das pessoas apresentarem um nível de colesterol menor
  após serem submetidas à dieta. Qual é a probabilidade de que o
  nutricionista endosse a nova dieta se, na verdade, ela não tem efeito
  algum sobre o nível de colesterol? (Admita que se a dieta não tem
  efeito algum sobre a quantidade de colesterol, então o nível de
  colesterol de cada pessoa será menor após a dieta com probabilidade
  1/2.)
\item
  Suponha que um componente eletrônico tenha um tempo de vida \(X\) (em
  unidades de 1000 horas) que é considerado uma variável aleatória com
  função densidade de probabilidade \(f(x) = e^{-x}\), \(x > 0\).
  Suponha também que o custo de fabricação de um item seja R\$ 2,00 e o
  preço de venda seja R\$ 5,00. O fabricante garante devolução total do
  dinheiro se \(x \leq 0,9\). Qual o lucro esperado do fabricante por
  item produzido?
\item
  Um ponto é escolhido ao acaso em um segmento de reta de comprimento
  \(L\). Determine a probabilidade de que a razão entre o menor e o
  maior segmentos obtidos após a escolha deste ponto seja menor que 1/4.
\item
  Um ônibus viaja entre duas cidades, A e B, separadas por uma distância
  de 100 quilômetros. Se o ônibus sofre uma avaria durante a viagem, a
  distância do local da avaria à cidade A tem uma distribuição uniforme
  no intervalo \((0, 100)\). Existe uma estação de reparo na cidade A,
  uma estação de reparo na cidade B, e uma outra estação no centro da
  rota entre A e B. Foi sugerido que seria mais eficiente ter as três
  estações de reparo localizadas a 25, 50 e 75 quilômetros,
  respectivamente, de A. Você concorda com esta sugestão? Justifique.
\item
  A confiabilidade de um mecanismo eletrônico é a probabilidade de que
  ele funcione sob as condições para as quais foi planejado. Uma amostra
  de 1000 destes itens é escolhida ao acaso, e estes itens selecionados
  são testados. Calcule a probabilidade de se obter pelo menos 30 itens
  defeituosos, supondo que a confiabilidade de cada item é 0,95. Indique
  que suposições está fazendo ao efetuar tal cálculo.
\item
  A quantidade de chuva anual em uma certa região é normalmente
  distribuída com \(\mu = 40\) e \(\sigma = 4\). Qual a probabilidade de
  que, começando a registrar os índices pluviométricos este ano, serão
  necessários mais do que 10 anos para se registrar uma quantidade de
  chuva anual maior que 50? Que suposições você está fazendo?
\item
  Um corpo de bombeiros está para ser construído ao longo de uma estrada
  de comprimento \(A\), \(A < \infty\). Se incêndios ocorrem de maneira
  uniforme ao longo desta estrada, onde o corpo de bombeiros deveria ser
  instalado para minimizar a distância esperada ao incêndio? Isto é,
  determine \(\alpha\) que minimize \(E[|X - \alpha|]\), onde \(X\), que
  denota a posição da ocorrência de um incêndio nesta estrada, é
  uniformemente distribuído em \((0, A)\).
\item
  O tempo (em horas) necessário para reparar uma máquina é uma variável
  aleatória exponencialmente distribuída com parâmetro
  \(\lambda = \frac{1}{2}\). Determine:

  \begin{enumerate}
  \def\labelenumii{(\alph{enumii})}
  \tightlist
  \item
    A probabilidade de que o tempo de reparo exceda duas horas.
  \item
    A probabilidade condicional de que o tempo de reparo será maior que
    onze horas dado que a duração do reparo excede nove horas.
  \end{enumerate}
\item
  Uma empresa produz automóveis e garante a restituição da quantia paga
  se qualquer automóvel apresentar algum defeito grave no prazo de seis
  meses. A empresa produz automóveis do tipo A comum e do tipo B de
  luxo, com um lucro de R\$ 1.000,00 e R\$ 2.000,00, respectivamente,
  caso não haja restituição, e com um prejuízo de R\$ 3.000,00 e R\$
  8.000,00, respectivamente, se houver restituição. Suponha que o tempo
  para a ocorrência de algum defeito grave seja, em ambos os casos, uma
  variável aleatória com distribuição normal, respectivamente com médias
  9 meses e 12 meses, e variâncias 4 meses e 9 meses. Se tivesse que
  planejar uma estratégia de marketing para a empresa, você incentivaria
  as vendas dos automóveis do tipo A ou do tipo B. Justifique.
\item
  Uma máquina funciona se pelo menos três de cinco dos seus componentes
  estiverem funcionando. Cada componente, independentemente dos demais,
  funciona por um tempo que é uma variável aleatória com função
  densidade de probabilidade dada por \(f(x) = \frac{1}{5}e^{-x/5}\),
  \(x > 0\), e em horas. (a) Determine a probabilidade de que a máquina
  funcione por mais de cinco horas. (b) Determine o número médio de
  componentes funcionando dez horas após a máquina ser ligada. Deixe
  claras as suposições que está fazendo para resolver os itens (a) e
  (b).
\item
  Suponha que o número de horas \(X\) que uma máquina opera antes de
  falhar tem uma distribuição contínua com função densidade de
  probabilidade \(f(x) = \frac{1}{100}e^{-x/100}\), \(x > 0\). Suponha
  que no momento em que a máquina é colocada para funcionar, você deve
  decidir quando retornar para inspecioná-la. Se você retorna antes de a
  máquina falhar, você incorre num custo de B reais por ter perdido a
  inspeção. Se você retorna após a quebra da máquina, você incorre num
  custo de C reais por hora pelo período de tempo durante o qual a
  máquina não estava operando após a falha. Qual o número ótimo de horas
  que você deve esperar antes de retornar para a inspeção de forma a
  minimizar seu custo esperado?
\item
  Uma fração desconhecida \(p\) de uma certa população é de fumantes, e
  uma amostragem aleatória com reposição será usada para avaliar o valor
  de \(p\). Deseja-se encontrar \(p\) com um erro que não exceda 0,005,
  com alta probabilidade. Em outras palavras, deseja-se que, com alta
  probabilidade, a proporção amostral a ser obtida não difira mais de
  0,005 da proporção \(p\) real de fumantes. Qual deve ser o tamanho da
  amostra \(n\) para que isto ocorra? Explicite as suposições que está
  fazendo. Admitindo-se agora que \(p \leq 20\%\). Qual deve ser o
  tamanho da amostra \(n\) para que a condição exposta acima seja
  satisfeita?
\item
  Com base na definição de taxa de falha, determine a função taxa de
  falha de uma variável aleatória gama com parâmetros \(t\) e
  \(\lambda\).
\item
  Calcule a função taxa de falha de \(X\) quando \(X\) é uniformemente
  distribuída em \((0, a)\), \(a > 0\).
\item
  Costuma-se dizer que a taxa de mortalidade de pessoas fumantes é, em
  cada idade, o dobro da taxa de mortalidade de pessoas não-fumantes. O
  que isto significa? Isto significa que a probabilidade de uma pessoa
  não-fumante sobreviver um determinado número de anos corresponde a
  duas vezes a probabilidade de uma pessoa fumante, de mesma idade,
  sobreviver este mesmo número de anos?
\item
  Seja \(X\) uma variável aleatória não-negativa e contínua. Mostre que:
  \[ E(X) = \int_{0}^{\infty} P(X > t)dt \]
\item
  Para uma variável aleatória não-negativa \(T\), o tempo médio de vida
  residual é definido por \(m(t) = E[T - t | T \geq t]\).

  \begin{enumerate}
  \def\labelenumii{(\alph{enumii})}
  \tightlist
  \item
    Mostre que \(m(t) = \frac{\int_{t}^{\infty} R(x)dx}{R(t)}\).
    Interprete o significado de \(m(t)\).
  \item
    Calcule \(m(t)\) quando \(T\) tem distribuição exponencial de
    parâmetro \(\lambda\).
  \item
    Calcule \(m(t)\) quando \(T\) é uniformemente distribuído no
    intervalo \((0, A)\), \(A < \infty\).
  \end{enumerate}
\item
  Determine a função geradora de momentos de uma variável aleatória
  uniformemente distribuída no intervalo \((a, b)\).
\item
  Uma variável aleatória \(X\) tem distribuição beta com parâmetros
  \(\alpha\) e \(\beta\), \(\alpha, \beta > 0\), se sua função densidade
  de probabilidade é dada por
  \(f(x) = \frac{1}{B(\alpha,\beta)}x^{\alpha-1}(1-x)^{\beta-1}\),
  \(0 < x < 1\), e zero caso contrário, onde
  \(B(\alpha,\beta) = \int_{0}^{1} z^{\alpha-1}(1-z)^{\beta-1}dz = \frac{\Gamma(\alpha)\Gamma(\beta)}{\Gamma(\alpha+\beta)}\).

  \begin{enumerate}
  \def\labelenumii{(\alph{enumii})}
  \tightlist
  \item
    Determine \(E(X)\) e \(Var(X)\).
  \item
    Para \(\alpha > 1\) e \(\beta > 1\), determine \(x^*\) tal que
    \(f(x^*) = \max_{0<x<1} f(x)\). A distribuição beta é comumente
    utilizada em inferência Bayesiana quando se estudam quantidades de
    interesse assumindo valores no intervalo \((0, 1)\) como, por
    exemplo, a proporção de fumantes em uma certa população. Observe
    ainda que, se \(\alpha = \beta = 1\), \(X\) é uniformemente
    distribuída em \((0, 1)\).
  \end{enumerate}
\item
  Uma variável aleatória \(X\) tem distribuição de Laplace com parâmetro
  \(\lambda, \lambda > 0\), se sua função densidade de probabilidade é
  dada por:
  \[ f(x) = \frac{\lambda}{2}e^{-\lambda|x|}, x \in \mathbb{R} \]
  Determine a função de distribuição de \(X\).
\item
  Se \(X\) é uniformemente distribuída no intervalo \((0, 1)\), qual a
  distribuição de \(Y = -\log X\)?
\item
  Se \(X\) é uma variável aleatória exponencial com parâmetro
  \(\lambda\) e \(c > 0\) uma constante real, determine a distribuição
  de \(Y = cX\).
\item
  A mediana de uma variável aleatória contínua, cuja função de
  distribuição \(F\) é o valor real \(m\) tal que
  \(F(m) = \frac{1}{2}\). Determine a mediana da variável aleatória
  \(X\) se \(X\) é (a) Uniformemente distribuída no intervalo
  \((a, b)\). (b) Normal com parâmetros \(\mu\) e \(\sigma^2\). (c)
  Exponencial com parâmetro \(\lambda\).
\item
  A moda de uma variável aleatória contínua, cuja função densidade de
  probabilidade é \(f\), é o valor de \(x\) para o qual \(f\) atinge
  valor máximo. Calcule a moda da variável aleatória \(X\) nos casos
  (a), (b) e (c) do exercício anterior.
\item
  Mostre que se \(X\) é uma variável aleatória Weibull com parâmetros
  \(\alpha\) e \(\beta\), então \(Y = (\frac{X}{\beta})^\alpha\) é uma
  variável aleatória exponencial com parâmetro 1, e vice-versa.
\item
  Se \(X\) é uniformemente distribuída em \((a, b)\), qual variável
  aleatória que possui uma relação linear com \(X\) é uniformemente
  distribuída no intervalo \((0, 1)\)? Em outras palavras, determine
  \(\alpha, \beta \in \mathbb{R}\), tais que \(Y = \alpha X + \beta\)
  tem distribuição uniforme \((0, 1)\).
\item
  Seja \(X\) uma variável aleatória contínua com função de distribuição
  \(F\) qualquer, defina a variável aleatória \(Y\) por \(Y = F(X)\).
  Mostre que \(Y\) é uniformemente distribuída no intervalo \((0, 1)\).
  Este resultado é de grande importância em técnicas de simulação de
  variáveis aleatórias, uma vez que, para obtermos observações de uma
  variável aleatória contínua com função de distribuição \(F\) qualquer,
  basta gerarmos valores da distribuição uniforme \((0, 1)\) e, em
  seguida, aplicarmos \(F^{-1}\) a estes valores, onde \(F^{-1}\) é a
  função inversa de \(F\).
\item
  Se \(X\) é uniformemente distribuída no intervalo \((-1, 1)\),
  determine a função densidade de probabilidade da variável aleatória
  \(Y = |X|\).
\item
  Ache a distribuição de \(R = A\sin\theta\), onde \(A\) é uma constante
  fixada e \(\theta\) é distribuída uniformemente em
  \((-\frac{\pi}{2}, \frac{\pi}{2})\). Tal variável aleatória aparece na
  teoria de balística. Se um projétil é lançado de um certo ponto da
  superfície da Terra, formando um ângulo \(\theta\) em relação a esta
  superfície e com velocidade \(v\), então o alcance deste lançamento,
  \(R\), pode ser expresso por \(R = \frac{v^2}{g}\sin(2\theta)\), onde
  \(g\) é a constante gravitacional.
\item
  Dizemos que uma variável aleatória \(X\) tem distribuição de Pareto
  com parâmetros \(\alpha\) e \(\beta\), \(\alpha, \beta > 0\), se sua
  função densidade de probabilidade é dada por:
  \(f(x) = \frac{\alpha}{\beta}(\frac{\beta}{x})^{\alpha+1}\),
  \(x > \beta\), \(f(x) = 0\), \(x < \beta\). Mostre que, para
  \(\alpha > 1\), \(E(X) = \frac{\alpha\beta}{\alpha-1}\) e para
  \(\alpha > 2\),
  \(Var(X) = \frac{\alpha\beta^2}{(\alpha-1)^2(\alpha-2)}\).
\item
  Outra distribuição bastante comum em economia é a distribuição
  lognormal. A variável aleatória \(X\) assumindo valores positivos tem
  uma distribuição lognormal com parâmetros \(\mu\) e \(\sigma^2\),
  \(\mu \in \mathbb{R}\) e \(\sigma^2 > 0\), se \(Y = \log X\) tem
  distribuição normal com média \(\mu\) e variância \(\sigma^2\).
  Determine: (a) A função densidade de probabilidade de \(X\). (b)
  \(E(X)\) e \(Var(X)\).
\item
  Seja \(X\) uma variável aleatória normal com média 0 e variância 1.
  Seja \(Y = X^2\). (a) Determine a função densidade de probabilidade de
  \(Y\). (b) Calcule \(E(Y)\) e \(Var(Y)\). \(Y\) construída deste modo
  é denominada uma variável aleatória com distribuição qui-quadrado
  (\(\chi^2\)) com 1 grau de liberdade. De um modo geral pode-se
  construir variáveis aleatórias \(\chi_n^2\) com \(n\) graus de
  liberdade, cujo papel é fundamental em diversos métodos estatísticos.
\item
  Seja \(X\) uma variável com distribuição gama com parâmetros
  \(\alpha\) e \(\beta\). Defina a variável aleatória \(Z\) como sendo
  \(Z = 2\beta X\). Determine a função densidade de probabilidade de
  \(Z\), com \(2\alpha\) e \(\frac{1}{2}\). Qual a distribuição de \(Z\)
  se \(\alpha = \frac{n}{2}\)?
\item
  Seja \(X\) uma variável aleatória com distribuição beta de parâmetros
  \(\alpha\) e \(\beta\). Defina a variável aleatória \(Y\) como sendo
  \(Y = \frac{\alpha X}{\beta(1 - X)}\). Obtenha a função densidade de
  probabilidade de \(Y\). (A variável aleatória \(Y\) construída deste
  modo possui distribuição denominada distribuição F-Snedecor com
  \(2\alpha\) e \(2\beta\) graus de liberdade. Como a distribuição de
  qui-quadrado, a distribuição F-Snedecor também é de extrema
  importância em diversos métodos estatísticos.)
\item
  Seja \(X\) uma variável aleatória contínua com função densidade de
  probabilidade \(f\), que depende apenas de \(x\) e de um parâmetro
  \(\theta\), isto é,

  \begin{enumerate}
  \def\labelenumii{\arabic{enumii})}
  \tightlist
  \item
    \(f(x) \geq 0, \forall x \in \mathbb{R}\);
  \item
    \(\int_{-\infty}^{\infty} f(x)dx = 1\);
  \item
    \(f(x) = f(x, \theta)\). Dizemos que \(f\) pertence à família
    exponencial uniparamétrica de distribuições se \(f\) pode ser
    escrita da seguinte forma:
    \(f(X) = f(x, \theta) = S(x)h(\theta)e^{c(\theta)T(x)}\),
    \(\forall x \in D\), onde \(D\) não depende de \(\theta\) e
    \(S, T, h\) e \(c\) são funções reais. Verifique quais das
    distribuições a seguir pertencem à família exponencial: (a)
    Distribuição uniforme no intervalo \((0, A)\), \(A < \infty\). (b)
    Distribuição normal com média \(\mu\) e variância \(A\). (c)
    Distribuição exponencial com parâmetro \(\lambda, \lambda > 0\). (d)
    Distribuição beta com parâmetros \(\alpha\) e \(\beta\), \(\alpha\)
    conhecido.
  \end{enumerate}
\end{enumerate}

\end{document}
