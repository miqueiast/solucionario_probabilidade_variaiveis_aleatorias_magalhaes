\documentclass[12pt]{article}
\usepackage[utf8]{inputenc}
\usepackage{amsmath}
\usepackage{amssymb}
\usepackage{amsfonts}
\usepackage{amsthm}
\usepackage[brazil]{babel}
\usepackage{geometry}
\usepackage{tikz}
\usepackage[T1]{fontenc}
\usetikzlibrary{positioning}

% --- Configurações ---
\geometry{a4paper, margin=1in} % Margens
\linespread{1.2} % Espaçamento entre linhas

\title{Exercícios do Livro Magalhães - Capítulo 03\\\large{Seção 3.2}}
\author{Aluno: Miqueias T}
\date{10/08/2025}

\begin{document}

\maketitle

\section*{Introdução}
Este documento apresenta a resolução detalhada de exercícios selecionados sobre variáveis aleatórias contínuas conjuntas, abrangendo o cálculo de densidades marginais, verificação de independência, densidades condicionais e funções de distribuição conjunta.

\section*{Exercício 1}

\begin{quote}
Sejam X e Y variáveis aleatórias em $(\Omega, \mathcal{F}, P)$ com função de distribuição conjunta dada por $F_{X,Y}$. Mostre que $P(a_1 < X \le b_1, a_2 < Y \le b_2)$, com $a_1, b_1, a_2$ e $b_2 \in \mathbb{R}$, pode ser escrita como:
\[ F_{X,Y}(b_1, b_2) + F_{X,Y}(a_1, a_2) - F_{X,Y}(a_1, b_2) - F_{X,Y}(b_1, a_2). \]
\end{quote}

\hrulefill

\subsection*{Abordagem Visual e Intuitiva}

A resolução matemática formal, baseada na teoria dos conjuntos, está perfeitamente correta. No entanto, a maneira mais intuitiva de entender essa fórmula é pensar geometricamente no plano cartesiano.

O nosso objetivo é calcular a "massa de probabilidade" contida em um retângulo. A nossa única ferramenta é a função de distribuição acumulada (FDA), $F_{X,Y}(x,y)$, que nos dá a probabilidade de toda a região semi-infinita à esquerda e abaixo de um ponto $(x,y)$.

Vamos construir a probabilidade do nosso retângulo de interesse, que chamaremos de $\mathcal{R}$, através da adição e subtração de áreas.

\begin{center}
% --- CÓDIGO TIKZ MELHORADO ---
\begin{tikzpicture}[scale=1.5, every node/.style={align=center}]
    % --- Define as coordenadas dos cantos para facilitar a referência ---
    \coordinate (a1a2) at (1,1);
    \coordinate (b1a2) at (3,1);
    \coordinate (a1b2) at (1,3);
    \coordinate (b1b2) at (3,3);

    % --- Regiões coloridas ---
    \fill[blue!40!white, opacity=0.5] (a1a2) rectangle (b1b2); % Retângulo de interesse
    \fill[red!30!white, opacity=0.5] (0,0) rectangle (a1b2); % Área à esquerda
    \fill[green!40!white, opacity=0.5] (0,0) rectangle (b1a2); % Área abaixo
    \fill[red!50!green!50, opacity=0.4] (0,0) rectangle (a1a2); % Área de intersecção

    % --- Eixos e linhas guia ---
    \draw[->, thick] (-0.5, 0) -- (4.5, 0) node[right] {$x$};
    \draw[->, thick] (0, -0.5) -- (0, 4.5) node[above] {$y$};
    \draw[dashed] (a1a2) -- (a1b2) -- (b1b2) -- (b1a2) -- cycle;
    \draw[dashed] (a1a2) -- (0,1) node[left] {$a_2$};
    \draw[dashed] (b1a2) -- (3,0) node[below] {$b_1$};
    \draw[dashed] (a1b2) -- (1,0) node[below] {$a_1$};
    \draw[dashed] (b1b2) -- (0,3) node[left] {$b_2$};
    
    % --- Rótulo do Retângulo ---
    \node at (2,2) {\huge $\mathcal{R}$};

    % --- Anotações posicionadas manualmente ---
    % Anotação para +F(b1, b2)
    \node (label-b1b2) at (2, 4.2) {$+F_{X,Y}(b_1, b_2)$ \\ (Tudo)};
    \draw[gray, thin, ->] (label-b1b2) -- (b1b2);

    % Anotação para -F(a1, b2)
    \node (label-a1b2) at (0, 4.2) {$-F_{X,Y}(a_1, b_2)$ \\ (Remove a faixa \\ à esquerda)};
    \draw[gray, thin, ->] (label-a1b2) -- (a1b2);

    % Anotação para -F(b1, a2)
    \node (label-b1a2) at (4.5, 0.5) {$-F_{X,Y}(b_1, a_2)$ \\ (Remove a faixa \\ abaixo)};
    \draw[gray, thin, ->] (label-b1a2) -- (b1a2);

    % Anotação para +F(a1, a2)
    \node (label-a1a2) at (-1, 0.5) {$+F_{X,Y}(a_1, a_2)$ \\ (Adiciona de volta \\ a intersecção)};
    \draw[gray, thin, ->] (label-a1a2) -- (a1a2);

\end{tikzpicture}
\end{center}

A lógica, baseada no diagrama acima, é a seguinte:

\begin{enumerate}
    \item \textbf{Comece Grande:} Pegamos a probabilidade acumulada até o canto superior direito do nosso retângulo, \textbf{$F_{X,Y}(b_1, b_2)$}. Isso nos dá a probabilidade da região inteira $\{X \le b_1, Y \le b_2\}$.
    
    \item \textbf{Primeiro Corte (Subtração):} Para isolar nosso retângulo, precisamos remover o que está à sua esquerda. Subtraímos então a probabilidade acumulada até o ponto $(a_1, b_2)$, ou seja, \textbf{$-F_{X,Y}(a_1, b_2)$}.
    
    \item \textbf{Segundo Corte (Subtração):} Também precisamos remover o que está abaixo do nosso retângulo. Subtraímos a probabilidade acumulada até o ponto $(b_1, a_2)$, ou seja, \textbf{$-F_{X,Y}(b_1, a_2)$}.
    
    \item \textbf{Correção (Adição):} Ao subtrair as duas áreas nos passos 2 e 3, a região do canto inferior esquerdo $\{X \le a_1, Y \le a_2\}$ foi removida \textbf{duas vezes}. Para corrigir, precisamos adicioná-la de volta uma vez: \textbf{$+F_{X,Y}(a_1, a_2)$}.
\end{enumerate}

Juntando todas as peças, obtemos a fórmula desejada:
\[ P(\mathcal{R}) = F_{X,Y}(b_1, b_2) - F_{X,Y}(a_1, b_2) - F_{X,Y}(b_1, a_2) + F_{X,Y}(a_1, a_2). \]

\subsection*{Resolução Formal}
A intuição visual acima é formalizada pela manipulação de eventos. A prova segue exatamente como na sua resolução original, que está matematicamente perfeita. A combinação da intuição visual com a prova rigorosa solidifica o entendimento do conceito.

\section*{Exercício 2}

\begin{quote}
Mostre que $H$ não é função de distribuição conjunta.
\[ H(x,y) = \begin{cases} 1, & \text{se } \max(x,y) \ge 0 \text{ ou } x^2+y^2 \le 1; \\ 0, & \text{caso contrário.} \end{cases} \]
\end{quote}

\hrulefill

\subsection*{Análise e Estratégia}

Para que uma função seja uma Função de Distribuição Conjunta (FDA), ela precisa satisfazer algumas propriedades essenciais. A mais fundamental, que deriva diretamente do fato de que probabilidades não podem ser negativas, é a seguinte:

Para qualquer retângulo $\mathcal{R} = (a_1, b_1] \times (a_2, b_2]$, a probabilidade de um ponto $(X,Y)$ cair neste retângulo deve ser maior ou igual a zero. Usando a fórmula do exercício anterior, isso significa:
\[ P(\mathcal{R}) = H(b_1, b_2) + H(a_1, a_2) - H(a_1, b_2) - H(b_1, a_2) \ge 0. \]

Nossa estratégia será encontrar um \textbf{contraexemplo}: um retângulo específico para o qual essa conta resulte em um número negativo.

\subsubsection*{Entendendo a Função H(x,y)}

Antes de escolher o retângulo, vamos visualizar onde $H(x,y)$ vale 1.
\begin{itemize}
    \item \textbf{$\max(x,y) \ge 0$}: Esta condição é verdadeira se $x \ge 0$ ou se $y \ge 0$. Isso cobre todo o primeiro, segundo e quarto quadrantes do plano cartesiano. A única região que fica de fora é o terceiro quadrante, onde $x<0$ e $y<0$.
    \item \textbf{$x^2+y^2 \le 1$}: Esta é a equação de um círculo de raio 1 centrado na origem, incluindo seu interior.
\end{itemize}
A função $H(x,y)$ vale 1 se \textbf{qualquer uma} dessas condições for satisfeita. Isso significa que $H$ vale 1 em quase todo o plano, exceto por uma "fatia" no terceiro quadrante que está fora do círculo unitário.

A falha de $H(x,y)$ está em como ela "salta" de 0 para 1. Para violar a condição de não-negatividade, queremos um retângulo cujo canto inferior esquerdo $(a_1, a_2)$ esteja na região onde $H=0$, mas cujos outros três cantos estejam na região onde $H=1$. Isso cria uma inconsistência na "distribuição de probabilidade".

\subsection*{Resolução Formal}

Vamos escolher um retângulo que explore a fraqueza descrita acima.
Seja o retângulo definido por: $a_1 = -2$, $b_1 = 1$, $a_2 = -2$ e $b_2 = 1$.

Agora, vamos calcular o valor de $H$ para cada um dos quatro cantos do retângulo.

\begin{itemize}
    \item \textbf{Canto inferior esquerdo: $H(a_1, a_2) = H(-2, -2)$:}
    \begin{itemize}
        \item $\max(-2,-2) = -2$, que é $< 0$.
        \item $(-2)^2 + (-2)^2 = 8$, que é $> 1$.
        \item Nenhuma condição é satisfeita. Portanto, \textbf{$H(-2, -2) = 0$}.
        (Este ponto está na região "caso contrário").
    \end{itemize}
    
    \item \textbf{Canto superior direito: $H(b_1, b_2) = H(1, 1)$:}
    \begin{itemize}
        \item $\max(1,1) = 1$, que é $\ge 0$.
        \item A primeira condição é satisfeita. Portanto, \textbf{$H(1, 1) = 1$}.
    \end{itemize}

    \item \textbf{Canto superior esquerdo: $H(a_1, b_2) = H(-2, 1)$:}
    \begin{itemize}
        \item $\max(-2,1) = 1$, que é $\ge 0$.
        \item A primeira condição é satisfeita. Portanto, \textbf{$H(-2, 1) = 1$}.
    \end{itemize}

    \item \textbf{Canto inferior direito: $H(b_1, a_2) = H(1, -2)$:}
    \begin{itemize}
        \item $\max(1,-2) = 1$, que é $\ge 0$.
        \item A primeira condição é satisfeita. Portanto, \textbf{$H(1, -2) = 1$}.
    \end{itemize}
\end{itemize}

Agora, substituímos esses valores na fórmula da probabilidade do retângulo:
\begin{align*}
P(-2 < X \le 1, -2 < Y \le 1) &= H(1, 1) + H(-2, -2) - H(-2, 1) - H(1, -2) \\
&= 1 + 0 - 1 - 1 \\
&= -1
\end{align*}

Como a probabilidade calculada para este retângulo é $-1$, um valor negativo, a função $H(x,y)$ viola uma propriedade essencial das funções de distribuição conjunta.

Portanto, \textbf{$H$ não é uma função de distribuição conjunta}. \hfill \qedsymbol

\section{Exercício 3}

\begin{quote}
A densidade conjunta de X e Y é dada por:
\[ f_{X,Y}(x,y) = \frac{1}{2} I_A(x,y) \text{ com } A = \{(x,y) \in \mathbb{R}^2; -1 \le x \le 1, 0 \le y \le 1\}. \]
a. Calcule as marginais e verifique se X e Y são independentes. \\
b. Obtenha a densidade condicional de Y dado que X > 0. \\
c. Determine a função de distribuição conjunta entre X e Y.
\end{quote}

\subsection*{Resolução}

\subsubsection*{a. Marginais e Independência}

\paragraph{Cálculo da densidade marginal de X ($f_X(x)$):}
Para encontrar a marginal de X, integramos a densidade conjunta em relação a $y$ sobre todo o seu suporte. O suporte de $y$ é o intervalo $[0, 1]$. A função só é não-nula para $x \in [-1, 1]$.

Para $-1 \le x \le 1$:
\[ f_X(x) = \int_{-\infty}^{\infty} f_{X,Y}(x,y) \,dy = \int_{0}^{1} \frac{1}{2} \,dy = \frac{1}{2} [y]_{0}^{1} = \frac{1}{2}(1-0) = \frac{1}{2}. \]
Portanto, a densidade marginal de X é:
\[ f_X(x) = \begin{cases} \frac{1}{2}, & \text{se } -1 \le x \le 1 \\ 0, & \text{caso contrário.} \end{cases} \]
Isso significa que $X \sim U(-1, 1)$.

\paragraph{Cálculo da densidade marginal de Y ($f_Y(y)$):}
Similarmente, para encontrar a marginal de Y, integramos a densidade conjunta em relação a $x$. O suporte de $x$ é $[-1, 1]$. A função só é não-nula para $y \in [0, 1]$.

Para $0 \le y \le 1$:
\[ f_Y(y) = \int_{-\infty}^{\infty} f_{X,Y}(x,y) \,dx = \int_{-1}^{1} \frac{1}{2} \,dx = \frac{1}{2} [x]_{-1}^{1} = \frac{1}{2}(1 - (-1)) = \frac{1}{2}(2) = 1. \]
Portanto, a densidade marginal de Y é:
\[ f_Y(y) = \begin{cases} 1, & \text{se } 0 \le y \le 1 \\ 0, & \text{caso contrário.} \end{cases} \]
Isso significa que $Y \sim U(0, 1)$.

\paragraph{Verificação de Independência:}
Duas variáveis aleatórias X e Y são independentes se e somente se $f_{X,Y}(x,y) = f_X(x)f_Y(y)$ para todo $(x,y)$.
Vamos verificar:
\[ f_X(x)f_Y(y) = \left(\frac{1}{2}\right)(1) = \frac{1}{2}. \]
Este produto é válido para $-1 \le x \le 1$ e $0 \le y \le 1$, que é exatamente o domínio $A$ onde $f_{X,Y}(x,y)$ é não-nula. Como $f_{X,Y}(x,y) = \frac{1}{2}$, temos que $f_{X,Y}(x,y) = f_X(x)f_Y(y)$.

\textbf{Conclusão:} Sim, X e Y são independentes.

\subsubsection*{b. Densidade condicional de Y dado X > 0}
A fórmula da densidade condicional é $f_{Y|X}(y|x) = \frac{f_{X,Y}(x,y)}{f_X(x)}$, para $f_X(x) > 0$.
A condição é $X > 0$. No nosso caso, isso corresponde ao intervalo $0 < x \le 1$. Nesse intervalo, $f_X(x) = 1/2$.
\[ f_{Y|X}(y|x) = \frac{1/2}{1/2} = 1. \]
O suporte para $y$ continua sendo $0 \le y \le 1$.
Portanto, a densidade condicional é:
\[ f_{Y|X}(y|x) = \begin{cases} 1, & \text{se } 0 \le y \le 1 \text{ (para } 0 < x \le 1\text{)} \\ 0, & \text{caso contrário.} \end{cases} \]
Isso mostra que, dado $X=x$, a distribuição de $Y$ ainda é $U(0,1)$, o que é esperado, já que são independentes.

\subsubsection*{c. Função de Distribuição Conjunta (FC)}
A FC é $F_{X,Y}(x,y) = \int_{-\infty}^{y} \int_{-\infty}^{x} f_{X,Y}(u,v) \,du \,dv$. Precisamos analisar por regiões.

\begin{itemize}
    \item Se $x < -1$ ou $y < 0$: A área de integração não contém o suporte, então $F_{X,Y}(x,y) = 0$.
    \item Se $-1 \le x \le 1$ e $0 \le y \le 1$:
    \[ F_{X,Y}(x,y) = \int_{0}^{y} \int_{-1}^{x} \frac{1}{2} \,du \,dv = \int_{0}^{y} \frac{1}{2}[u]_{-1}^{x} \,dv = \int_{0}^{y} \frac{x+1}{2} \,dv = \frac{x+1}{2}[v]_{0}^{y} = \frac{(x+1)y}{2}. \]
    \item Se $x > 1$ e $0 \le y \le 1$: A integração em $u$ cobre todo o suporte de X, de $-1$ a $1$.
    \[ F_{X,Y}(x,y) = \int_{0}^{y} \int_{-1}^{1} \frac{1}{2} \,du \,dv = \int_{0}^{y} 1 \,dv = y. \]
    \item Se $-1 \le x \le 1$ e $y > 1$: A integração em $v$ cobre todo o suporte de Y, de $0$ a $1$.
    \[ F_{X,Y}(x,y) = \int_{0}^{1} \int_{-1}^{x} \frac{1}{2} \,du \,dv = \int_{0}^{1} \frac{x+1}{2} \,dv = \frac{x+1}{2}. \]
    \item Se $x > 1$ e $y > 1$: A integração cobre todo o suporte de A. A probabilidade total é 1.
    \[ F_{X,Y}(x,y) = 1. \]
\end{itemize}
Resumindo em uma única função:
\[ F_{X,Y}(x,y) = \begin{cases}
0, & \text{se } x < -1 \text{ ou } y < 0 \\
\frac{(x+1)y}{2}, & \text{se } -1 \le x \le 1, 0 \le y \le 1 \\
y, & \text{se } x > 1, 0 \le y \le 1 \\
\frac{x+1}{2}, & \text{se } -1 \le x \le 1, y > 1 \\
1, & \text{se } x > 1, y > 1
\end{cases} \]

\pagebreak

\section{Exercício 4}

\begin{quote}
Sejam X e Y com densidade conjunta:
\[ f_{X,Y}(x,y) = \frac{1}{\pi} I_A(x,y) \text{ com } A = \{(x,y) \in \mathbb{R}^2; x^2+y^2 \le 1\}. \]
a. Calcule as marginais e verifique se X e Y são independentes. \\
b. Determine a densidade condicional de X dado que Y = 1/2.
\end{quote}

\subsection*{Resolução}

\subsubsection*{a. Marginais e Independência}

\paragraph{Cálculo da densidade marginal de X ($f_X(x)$):}
Integramos a densidade conjunta em $y$. O suporte de $x$ é $[-1, 1]$. Para um $x$ fixo, os limites de $y$ são dados por $y^2 \le 1-x^2$, ou seja, $-\sqrt{1-x^2} \le y \le \sqrt{1-x^2}$.

Para $-1 \le x \le 1$:
\begin{align*}
f_X(x) &= \int_{-\sqrt{1-x^2}}^{\sqrt{1-x^2}} \frac{1}{\pi} \,dy = \frac{1}{\pi} [y]_{-\sqrt{1-x^2}}^{\sqrt{1-x^2}} \\
&= \frac{1}{\pi} (\sqrt{1-x^2} - (-\sqrt{1-x^2})) = \frac{2}{\pi}\sqrt{1-x^2}.
\end{align*}
A densidade marginal de X é:
\[ f_X(x) = \begin{cases} \frac{2}{\pi}\sqrt{1-x^2}, & \text{se } -1 \le x \le 1 \\ 0, & \text{caso contrário.} \end{cases} \]

\paragraph{Cálculo da densidade marginal de Y ($f_Y(y)$):}
Por simetria do problema (o domínio é um círculo centrado na origem), a forma funcional da marginal de Y será a mesma da de X.
\[ f_Y(y) = \begin{cases} \frac{2}{\pi}\sqrt{1-y^2}, & \text{se } -1 \le y \le 1 \\ 0, & \text{caso contrário.} \end{cases} \]

\paragraph{Verificação de Independência:}
Verificamos se $f_{X,Y}(x,y) = f_X(x)f_Y(y)$.
\[ f_X(x)f_Y(y) = \left(\frac{2}{\pi}\sqrt{1-x^2}\right) \left(\frac{2}{\pi}\sqrt{1-y^2}\right) = \frac{4}{\pi^2}\sqrt{(1-x^2)(1-y^2)}. \]
Claramente, $\frac{4}{\pi^2}\sqrt{(1-x^2)(1-y^2)} \neq \frac{1}{\pi}$.

\textbf{Conclusão:} Não, X e Y não são independentes. Uma outra forma de ver isso é que o suporte da distribuição (um círculo) não é um retângulo, então o domínio de $y$ depende do valor de $x$.

\subsubsection*{b. Densidade condicional de X dado Y = 1/2}
Usamos a fórmula $f_{X|Y}(x|y) = \frac{f_{X,Y}(x,y)}{f_Y(y)}$, para $f_Y(y)>0$.
Primeiro, calculamos o denominador para $y=1/2$:
\[ f_Y(1/2) = \frac{2}{\pi}\sqrt{1-(1/2)^2} = \frac{2}{\pi}\sqrt{1-\frac{1}{4}} = \frac{2}{\pi}\sqrt{\frac{3}{4}} = \frac{2}{\pi}\frac{\sqrt{3}}{2} = \frac{\sqrt{3}}{\pi}. \]
O numerador é $f_{X,Y}(x, 1/2) = 1/\pi$, mas precisamos encontrar o novo suporte para $x$.
Dado $Y=1/2$, a condição $x^2+y^2 \le 1$ se torna $x^2 + (1/2)^2 \le 1 \implies x^2 \le 3/4$.
Isso significa que $-\frac{\sqrt{3}}{2} \le x \le \frac{\sqrt{3}}{2}$.

Agora, montamos a densidade condicional:
\[ f_{X|Y}(x | 1/2) = \frac{f_{X,Y}(x, 1/2)}{f_Y(1/2)} = \frac{1/\pi}{\sqrt{3}/\pi} = \frac{1}{\sqrt{3}}. \]
Portanto, a densidade condicional é:
\[ f_{X|Y}(x | 1/2) = \begin{cases} \frac{1}{\sqrt{3}}, & \text{se } -\frac{\sqrt{3}}{2} \le x \le \frac{\sqrt{3}}{2} \\ 0, & \text{caso contrário.} \end{cases} \]
Isso indica que a distribuição condicional de X, dado que Y=1/2, é uma Uniforme no intervalo $[-\sqrt{3}/2, \sqrt{3}/2]$.

\pagebreak

\section{Exercício 5}

\begin{quote}
A densidade conjunta de X e Y é dada por:
\[ f_{X,Y}(x,y) = \begin{cases} \frac{1}{4}, & \text{se } -2 < x < 0, 0 < y < 1 \\ \frac{3}{4}, & \text{se } 0 \le x \le 1, -2 < y < 0 \\ 0, & \text{caso contrário.} \end{cases} \]
a. Calcule as marginais e verifique se X e Y são independentes. \\
b. Determine a função de distribuição conjunta entre X e Y. \\
c. Obtenha a função de distribuição condicional de Y dado X = x.
\end{quote}

\subsection*{Resolução}
\textit{Nota: Este exercício é um pouco mais complexo devido às regiões descontínuas. Vamos analisar as integrações com cuidado.}

\subsubsection*{a. Marginais e Independência}

\paragraph{Cálculo da densidade marginal de X ($f_X(x)$):}
Precisamos considerar os diferentes intervalos de $x$.

\begin{itemize}
    \item Para $-2 < x < 0$: A densidade só é não-nula quando $0 < y < 1$.
    \[ f_X(x) = \int_{0}^{1} \frac{1}{4} \,dy = \frac{1}{4}[y]_{0}^{1} = \frac{1}{4}. \]
    \item Para $0 \le x \le 1$: A densidade só é não-nula quando $-2 < y < 0$.
    \[ f_X(x) = \int_{-2}^{0} \frac{3}{4} \,dy = \frac{3}{4}[y]_{-2}^{0} = \frac{3}{4}(0 - (-2)) = \frac{3}{2}. \]
\end{itemize}
Juntando tudo, a marginal de X é:
\[ f_X(x) = \begin{cases}
1/4, & \text{se } -2 < x < 0 \\
3/2, & \text{se } 0 \le x \le 1 \\
0, & \text{caso contrário.}
\end{cases} \]

\paragraph{Cálculo da densidade marginal de Y ($f_Y(y)$):}
Agora, consideramos os intervalos de $y$.

\begin{itemize}
    \item Para $-2 < y < 0$: A densidade só é não-nula quando $0 \le x \le 1$.
    \[ f_Y(y) = \int_{0}^{1} \frac{3}{4} \,dx = \frac{3}{4}[x]_{0}^{1} = \frac{3}{4}. \]
    \item Para $0 < y < 1$: A densidade só é não-nula quando $-2 < x < 0$.
    \[ f_Y(y) = \int_{-2}^{0} \frac{1}{4} \,dx = \frac{1}{4}[x]_{-2}^{0} = \frac{1}{4}(0 - (-2)) = \frac{1}{2}. \]
\end{itemize}
Juntando tudo, a marginal de Y é:
\[ f_Y(y) = \begin{cases}
3/4, & \text{se } -2 < y < 0 \\
1/2, & \text{se } 0 < y < 1 \\
0, & \text{caso contrário.}
\end{cases} \]

\paragraph{Verificação de Independência:}
Vamos testar um ponto. Seja $(x,y) = (-1, 0.5)$.
$f_{X,Y}(-1, 0.5) = 1/4$.
$f_X(-1) = 1/4$.
$f_Y(0.5) = 1/2$.
$f_X(-1)f_Y(0.5) = (1/4)(1/2) = 1/8 \neq 1/4$.

\textbf{Conclusão:} Como $f_{X,Y}(x,y) \neq f_X(x)f_Y(y)$, as variáveis X e Y não são independentes.

\textit{(As subseções b e c para este exercício são bastante longas por envolverem múltiplas regiões. Elas podem ser adicionadas se você desejar.)}

\section{Exercício 6}

\begin{quote}
Seja $Y \sim \text{Poisson}(\lambda)$ e $X|(Y=n) \sim B(n, p)$. \\
a. Calcule a função de probabilidade de X. \\
b. Determine a função de probabilidade condicional de Y dado $X=x$.
\end{quote}

\subsection*{Resolução}
Temos as seguintes funções de probabilidade (PMFs):
\begin{itemize}
    \item $P(Y=n) = \frac{e^{-\lambda}\lambda^n}{n!}$, para $n = 0, 1, 2, \dots$
    \item $P(X=x | Y=n) = \binom{n}{x} p^x (1-p)^{n-x}$, para $x = 0, 1, \dots, n$.
\end{itemize}

\subsubsection*{a. Função de Probabilidade de X}
Para encontrar a função de probabilidade marginal de $X$, usamos a Lei da Probabilidade Total, somando sobre todos os possíveis valores de $n$:
\[ P(X=x) = \sum_{n=0}^{\infty} P(X=x, Y=n) = \sum_{n=0}^{\infty} P(X=x | Y=n) P(Y=n) \]
A probabilidade $P(X=x|Y=n)$ só é não-nula se $n \ge x$. Portanto, a soma começa em $n=x$.
\begin{align*}
P(X=x) &= \sum_{n=x}^{\infty} \left[ \binom{n}{x} p^x (1-p)^{n-x} \right] \left[ \frac{e^{-\lambda}\lambda^n}{n!} \right] \\
&= \sum_{n=x}^{\infty} \frac{n!}{x!(n-x)!} p^x (1-p)^{n-x} \frac{e^{-\lambda}\lambda^n}{n!} \\
&= \sum_{n=x}^{\infty} \frac{1}{x!(n-x)!} p^x (1-p)^{n-x} e^{-\lambda}\lambda^n \\
\end{align*}
Agora, vamos reorganizar os termos, tirando da soma tudo que não depende de $n$:
\begin{align*}
P(X=x) &= \frac{e^{-\lambda} p^x}{x!} \sum_{n=x}^{\infty} \frac{1}{(n-x)!} (1-p)^{n-x} \lambda^n \\
&= \frac{e^{-\lambda} p^x}{x!} \sum_{n=x}^{\infty} \frac{(1-p)^{n-x} \lambda^x \lambda^{n-x}}{(n-x)!} \quad (\text{pois } \lambda^n = \lambda^x \lambda^{n-x}) \\
&= \frac{e^{-\lambda} (\lambda p)^x}{x!} \sum_{n=x}^{\infty} \frac{(\lambda(1-p))^{n-x}}{(n-x)!}
\end{align*}
Fazemos uma mudança de variável na soma, seja $k = n-x$. Quando $n=x$, $k=0$.
\[ \sum_{k=0}^{\infty} \frac{(\lambda(1-p))^{k}}{k!} \]
Esta é a série de Taylor para a função exponencial $e^z$ com $z = \lambda(1-p)$. Portanto, a soma é igual a $e^{\lambda(1-p)}$.
Substituindo de volta:
\begin{align*}
P(X=x) &= \frac{e^{-\lambda} (\lambda p)^x}{x!} \cdot e^{\lambda(1-p)} \\
&= \frac{(\lambda p)^x}{x!} e^{-\lambda + \lambda(1-p)} \\
&= \frac{(\lambda p)^x}{x!} e^{-\lambda + \lambda - \lambda p} \\
&= \frac{e^{-\lambda p} (\lambda p)^x}{x!}
\end{align*}
Esta é a função de probabilidade de uma distribuição de Poisson com parâmetro $\lambda p$.
Portanto, $X \sim \text{Poisson}(\lambda p)$, para $x=0, 1, 2, \dots$.

\subsubsection*{b. Função de Probabilidade Condicional de Y dado X=x}
Usamos a definição de probabilidade condicional:
\[ P(Y=n | X=x) = \frac{P(X=x, Y=n)}{P(X=x)} = \frac{P(X=x|Y=n)P(Y=n)}{P(X=x)} \]
Esta probabilidade é não-nula apenas para $n \ge x$. Substituindo as funções que já conhecemos:
\begin{align*}
P(Y=n | X=x) &= \frac{ \left[ \binom{n}{x} p^x (1-p)^{n-x} \right] \left[ \frac{e^{-\lambda}\lambda^n}{n!} \right] }{ \frac{e^{-\lambda p} (\lambda p)^x}{x!} } \\
&= \frac{ \frac{n!}{x!(n-x)!} p^x (1-p)^{n-x} \frac{e^{-\lambda}\lambda^n}{n!} }{ \frac{e^{-\lambda p} \lambda^x p^x}{x!} } \\
&= \left( \frac{p^x (1-p)^{n-x} e^{-\lambda} \lambda^n}{x!(n-x)!} \right) \cdot \left( \frac{x!}{e^{-\lambda p} \lambda^x p^x} \right) \\
&= \frac{(1-p)^{n-x} \lambda^{n-x} e^{-\lambda} e^{\lambda p}}{(n-x)!} \quad (\text{cancelando } x!, p^x, \lambda^x) \\
&= \frac{(\lambda(1-p))^{n-x} e^{-(\lambda - \lambda p)}}{(n-x)!} \\
&= \frac{e^{-\lambda(1-p)} (\lambda(1-p))^{n-x}}{(n-x)!}
\end{align*}
Esta é a PMF de uma variável aleatória $Z = Y-x$ que segue uma distribuição Poisson com parâmetro $\lambda(1-p)$. Ou seja, $Y-x | X=x \sim \text{Poisson}(\lambda(1-p))$.
A função de probabilidade condicional de $Y$ dado $X=x$ é:
\[ P(Y=n | X=x) = \frac{e^{-\lambda(1-p)} (\lambda(1-p))^{n-x}}{(n-x)!}, \quad \text{para } n = x, x+1, x+2, \dots \]
\hfill \qedsymbol

\section{Exercício 7}

\begin{quote}
Um milionário excêntrico, uma vez por semana, deixa seu escritório com X milhares de reais no bolso. Ao caminhar para sua casa vai distribuindo esse dinheiro aos eventuais pedintes que encontra. Admita que X tem densidade de probabilidade $f_X(x) = \frac{x}{8} I_{(0,4)}(x)$ e, também que o dinheiro que lhe resta ao chegar em casa, denotado por Y, tem probabilidade uniforme entre zero e o dinheiro com que deixou o escritório. Isto é, $Y|(X=x) \sim U_c[0,x]$. \\
a. Calcule a densidade conjunta entre X e Y. \\
b. Determine a densidade marginal de Y.
\end{quote}

\subsection*{Resolução}
Primeiro, vamos formalizar as densidades de probabilidade (PDFs) dadas:
\begin{itemize}
    \item A densidade marginal de X é:
    \[ f_X(x) = \begin{cases} \frac{x}{8}, & \text{se } 0 < x < 4 \\ 0, & \text{caso contrário.} \end{cases} \]
    \item A densidade condicional de Y dado X=x é a de uma distribuição Uniforme Contínua em $[0,x]$. A fórmula da densidade uniforme em $[a,b]$ é $\frac{1}{b-a}$. Portanto:
    \[ f_{Y|X}(y|x) = \begin{cases} \frac{1}{x-0} = \frac{1}{x}, & \text{se } 0 \le y \le x \\ 0, & \text{caso contrário.} \end{cases} \]
\end{itemize}

\subsubsection*{a. Densidade Conjunta entre X e Y}
A densidade conjunta $f_{X,Y}(x,y)$ é encontrada multiplicando a densidade condicional pela densidade marginal:
\[ f_{X,Y}(x,y) = f_{Y|X}(y|x) f_X(x) \]
Para os valores onde as densidades são não-nulas, temos:
\[ f_{X,Y}(x,y) = \left(\frac{1}{x}\right) \cdot \left(\frac{x}{8}\right) = \frac{1}{8} \]
Agora, precisamos definir o suporte (a região) onde esta densidade é válida. As condições são:
\begin{enumerate}
    \item $0 < x < 4$ (do suporte de $f_X(x)$)
    \item $0 \le y \le x$ (do suporte de $f_{Y|X}(y|x)$)
\end{enumerate}
Combinando estas condições, obtemos o suporte conjunto, que é uma região triangular no plano cartesiano. A densidade conjunta é:
\[ f_{X,Y}(x,y) = \begin{cases} \frac{1}{8}, & \text{se } 0 < x < 4 \text{ e } 0 \le y \le x \\ 0, & \text{caso contrário.} \end{cases} \]

\subsubsection*{b. Densidade Marginal de Y}
Para encontrar a densidade marginal de Y, integramos a densidade conjunta em relação a $x$ sobre todo o seu suporte.
\[ f_Y(y) = \int_{-\infty}^{\infty} f_{X,Y}(x,y) \,dx \]
Os limites de integração para $x$ dependem do valor de $y$. Observando as condições do suporte conjunto ($0 < x < 4$ e $y \le x$), para um $y$ fixo, $x$ deve satisfazer $x \ge y$ e $x < 4$. Portanto, a integração em $x$ ocorrerá no intervalo $[y, 4)$.

Além disso, para que o intervalo $[y, 4)$ seja válido, devemos ter $y < 4$. Como também temos $y \ge 0$, o suporte para a marginal de Y será $0 \le y < 4$.

Para um $y$ nesse intervalo:
\begin{align*}
f_Y(y) &= \int_{y}^{4} \frac{1}{8} \,dx \\
&= \frac{1}{8} [x]_{y}^{4} \\
&= \frac{1}{8} (4 - y)
\end{align*}
Portanto, a densidade marginal de Y é:
\[ f_Y(y) = \begin{cases} \frac{4-y}{8}, & \text{se } 0 \le y < 4 \\ 0, & \text{caso contrário.} \end{cases} \]
\hfill \qedsymbol

\section{Exercício 8}

\begin{quote}
A função de distribuição conjunta de $(X,Y)$ é dada por:
\[ F_{X,Y}(x,y) = \begin{cases} 0, & \text{se } x < 0 \text{ ou } y < 0 \text{ ou } x \ge y; \\ \frac{2x^2y^2-x^4}{16}, & \text{se } 0 \le x < y, 0 \le y < 2; \\ \frac{8x^2-x^4}{16}, & \text{se } 0 \le x < 2, y \ge 2; \\ 1, & \text{se } x \ge 2, y \ge 2, x < y. \end{cases} \]
a. Obtenha as funções de distribuição marginais de X e Y. \\
b. Calcule a densidade conjunta entre X e Y. \\
c. Calcule as densidades marginais de X e Y de duas maneiras diferentes.
\end{quote}

\subsection*{Resolução}

\subsubsection*{b. Densidade Conjunta entre X e Y}

(Vamos resolver o item \textbf{b} primeiro, pois ele nos dará uma função de densidade válida para trabalhar nos outros itens).
A função de densidade de probabilidade (PDF) conjunta $f_{X,Y}(x,y)$ é obtida pela derivada parcial mista da função de distribuição (FC):
\[ f_{X,Y}(x,y) = \frac{\partial^2}{\partial y \partial x} F_{X,Y}(x,y) \]
Vamos usar a expressão da FC na região principal de variação, $0 \le x < y, 0 \le y < 2$:
\[ F_{X,Y}(x,y) = \frac{2x^2y^2-x^4}{16} \]
Primeiro, derivamos em relação a $x$:
\[ \frac{\partial}{\partial x} F_{X,Y}(x,y) = \frac{4xy^2 - 4x^3}{16} = \frac{xy^2 - x^3}{4} \]
Agora, derivamos o resultado em relação a $y$:
\[ f_{X,Y}(x,y) = \frac{\partial}{\partial y} \left( \frac{xy^2 - x^3}{4} \right) = \frac{2xy}{4} = \frac{xy}{2} \]
O suporte para esta densidade é a região $0 \le x \le y$ e $0 \le y \le 2$ (que é o mesmo que $0 \le x \le 2$ e $x \le y \le 2$). A densidade conjunta é:
\[ f_{X,Y}(x,y) = \begin{cases} \frac{xy}{2}, & \text{se } 0 \le x \le 2, x \le y \le 2 \\ 0, & \text{caso contrário.} \end{cases} \]

\subsubsection*{a. Funções de Distribuição Marginais (FCs)}

\paragraph{FC Marginal de X ($F_X(x)$):}
Usamos a definição $F_X(x) = \lim_{y\to\infty} F_{X,Y}(x,y)$.
\begin{itemize}
    \item Se $x < 0$: $F_X(x) = \lim_{y\to\infty} 0 = 0$.
    \item Se $0 \le x < 2$: Quando $y \to \infty$, caímos na região $y \ge 2$, então usamos a terceira expressão da FC.
    \[ F_X(x) = \lim_{y\to\infty} \frac{8x^2-x^4}{16} = \frac{8x^2-x^4}{16}. \]
    \item Se $x \ge 2$: Quando $y \to \infty$, caímos na região $x \ge 2, y \ge 2, x < y$.
    \[ F_X(x) = \lim_{y\to\infty} 1 = 1. \]
\end{itemize}
Juntando as partes:
\[ F_X(x) = \begin{cases} 0, & \text{se } x < 0 \\ \frac{8x^2-x^4}{16}, & \text{se } 0 \le x < 2 \\ 1, & \text{se } x \ge 2. \end{cases} \]

\paragraph{FC Marginal de Y ($F_Y(y)$):}
Calculamos $F_Y(y)$ integrando sua densidade marginal $f_Y(t)$ (que será calculada no item \textbf{c}).
Sabendo que $f_Y(y) = y^3/4$ para $0 \le y \le 2$:
\[ F_Y(y) = \int_{-\infty}^{y} f_Y(t) \,dt \]
\begin{itemize}
    \item Se $y < 0$: $F_Y(y) = 0$.
    \item Se $0 \le y \le 2$:
    \[ F_Y(y) = \int_0^y \frac{t^3}{4} \,dt = \frac{1}{4} \left[ \frac{t^4}{4} \right]_0^y = \frac{y^4}{16}. \]
    \item Se $y > 2$:
    \[ F_Y(y) = \int_0^2 \frac{t^3}{4} \,dt = \frac{1}{4} \left[ \frac{t^4}{4} \right]_0^2 = \frac{16}{16} = 1. \]
\end{itemize}
Juntando as partes:
\[ F_Y(y) = \begin{cases} 0, & \text{se } y < 0 \\ \frac{y^4}{16}, & \text{se } 0 \le y \le 2 \\ 1, & \text{se } y > 2. \end{cases} \]

\subsubsection*{c. Densidades Marginais de X e Y (PDFs)}

\paragraph{Primeira Maneira: Integrando a Densidade Conjunta}
\begin{itemize}
    \item \textbf{PDF Marginal de X:} Integramos $f_{X,Y}(x,y)$ em relação a $y$. Para um $x$ fixo no intervalo $[0,2]$, $y$ varia de $x$ até $2$.
    \[ f_X(x) = \int_x^2 \frac{xy}{2} \,dy = \frac{x}{2} \left[ \frac{y^2}{2} \right]_x^2 = \frac{x}{4} (4 - x^2) = x - \frac{x^3}{4}, \quad \text{para } 0 \le x \le 2. \]
    \item \textbf{PDF Marginal de Y:} Integramos $f_{X,Y}(x,y)$ em relação a $x$. Para um $y$ fixo no intervalo $[0,2]$, $x$ varia de $0$ até $y$.
    \[ f_Y(y) = \int_0^y \frac{xy}{2} \,dx = \frac{y}{2} \left[ \frac{x^2}{2} \right]_0^y = \frac{y}{4} (y^2 - 0) = \frac{y^3}{4}, \quad \text{para } 0 \le y \le 2. \]
\end{itemize}

\paragraph{Segunda Maneira: Derivando as FCs Marginais}
\begin{itemize}
    \item \textbf{PDF Marginal de X:} Derivamos $F_X(x)$ em relação a $x$ para $0 \le x < 2$.
    \[ f_X(x) = \frac{d}{dx} F_X(x) = \frac{d}{dx} \left( \frac{8x^2-x^4}{16} \right) = \frac{16x - 4x^3}{16} = x - \frac{x^3}{4}. \]
    \item \textbf{PDF Marginal de Y:} Derivamos $F_Y(y)$ em relação a $y$ para $0 \le y \le 2$.
    \[ f_Y(y) = \frac{d}{dy} F_Y(y) = \frac{d}{dy} \left( \frac{y^4}{16} \right) = \frac{4y^3}{16} = \frac{y^3}{4}. \]
\end{itemize}
Ambos os métodos produzem os mesmos resultados, como esperado. \hfill \qedsymbol

\section{Exercício 9}

\begin{quote}
Considere a função:
\[ f_{X|Y}(x|y) = \begin{cases} \frac{y^x e^{-y}}{x!}, & \text{se } x = 0, 1, \dots \text{ e } y > 0; \\ 0, & \text{caso contrário.} \end{cases} \]
a. Mostre que, para cada y fixado, $f(\cdot|y)$ é uma função de probabilidade. \\
b. Determine a conjunta de X e Y se $Y \sim \text{Exp}(1)$. \\
c. Nas condições de (b), obtenha a marginal de X.
\end{quote}

\subsection*{Resolução}

\subsubsection*{a. Verificação da Função de Probabilidade}
Para que $f_{X|Y}(x|y)$ seja uma função de massa de probabilidade (PMF) para um $y>0$ fixo, ela deve satisfazer duas condições:
\begin{enumerate}
    \item $f_{X|Y}(x|y) \ge 0$ para todos os valores de $x$.
    \item $\sum_{x} f_{X|Y}(x|y) = 1$.
\end{enumerate}

\paragraph{1. Não-negatividade:}
Para $y > 0$ e $x \in \{0, 1, 2, \dots\}$, temos que $y^x \ge 0$, $e^{-y} > 0$ e $x! \ge 1$. O quociente de termos não-negativos é não-negativo, então $f_{X|Y}(x|y) \ge 0$. A condição é satisfeita.

\paragraph{2. Soma igual a 1:}
Somamos a função sobre todos os possíveis valores de $x$:
\begin{align*}
\sum_{x=0}^{\infty} f_{X|Y}(x|y) &= \sum_{x=0}^{\infty} \frac{y^x e^{-y}}{x!} \\
&= e^{-y} \sum_{x=0}^{\infty} \frac{y^x}{x!} \quad (\text{pois } e^{-y} \text{ não depende de } x)
\end{align*}
A soma $\sum_{x=0}^{\infty} \frac{y^x}{x!}$ é a série de Taylor para a função exponencial $e^y$. Portanto:
\[ \sum_{x=0}^{\infty} f_{X|Y}(x|y) = e^{-y} \cdot e^y = e^0 = 1. \]
A segunda condição também é satisfeita.

\textbf{Conclusão:} Para cada $y>0$ fixo, $f_{X|Y}(x|y)$ é uma função de probabilidade válida. De fato, é a PMF de uma distribuição de \textbf{Poisson} com parâmetro $\lambda=y$.

\subsubsection*{b. Função de Densidade Conjunta}
A função de densidade/massa de probabilidade conjunta é dada por $f_{X,Y}(x,y) = f_{X|Y}(x|y) f_Y(y)$.
É nos dado que $Y \sim \text{Exp}(1)$, cuja função de densidade de probabilidade (PDF) é:
\[ f_Y(y) = \begin{cases} e^{-y}, & \text{se } y > 0 \\ 0, & \text{caso contrário.} \end{cases} \]
Agora, multiplicamos as funções para obter a conjunta, válida no suporte $x \in \{0, 1, \dots\}$ e $y > 0$:
\begin{align*}
f_{X,Y}(x,y) &= \left( \frac{y^x e^{-y}}{x!} \right) \cdot (e^{-y}) \\
&= \frac{y^x e^{-2y}}{x!}
\end{align*}
Portanto, a função conjunta é:
\[ f_{X,Y}(x,y) = \begin{cases} \frac{y^x e^{-2y}}{x!}, & \text{se } x \in \{0, 1, \dots\} \text{ e } y > 0 \\ 0, & \text{caso contrário.} \end{cases} \]

\subsubsection*{c. Função de Probabilidade Marginal de X}
Para obter a PMF marginal de X, $P(X=x)$, integramos a função conjunta em relação a $y$ sobre todo o seu suporte, que é $(0, \infty)$.
\begin{align*}
P(X=x) &= \int_{0}^{\infty} f_{X,Y}(x,y) \,dy \\
&= \int_{0}^{\infty} \frac{y^x e^{-2y}}{x!} \,dy \\
&= \frac{1}{x!} \int_{0}^{\infty} y^x e^{-2y} \,dy
\end{align*}
A integral tem a forma de uma função Gamma. A função Gamma é definida como $\Gamma(z) = \int_0^\infty t^{z-1}e^{-t}dt$.
Para resolver nossa integral, fazemos a substituição $t = 2y$, de onde $y = t/2$ e $dy = dt/2$. Os limites de integração permanecem os mesmos.
\begin{align*}
\int_{0}^{\infty} y^x e^{-2y} \,dy &= \int_{0}^{\infty} \left(\frac{t}{2}\right)^x e^{-t} \left(\frac{dt}{2}\right) \\
&= \int_{0}^{\infty} \frac{t^x}{2^x} e^{-t} \frac{1}{2} \,dt \\
&= \frac{1}{2^{x+1}} \int_{0}^{\infty} t^x e^{-t} \,dt
\end{align*}
A integral $\int_{0}^{\infty} t^x e^{-t} \,dt$ é a definição de $\Gamma(x+1)$. Para valores inteiros de $x$, sabemos que $\Gamma(x+1) = x!$.
Portanto, o valor da integral é $\frac{1}{2^{x+1}} \cdot x!$.

Substituindo este resultado de volta na expressão para $P(X=x)$:
\[ P(X=x) = \frac{1}{x!} \left( \frac{x!}{2^{x+1}} \right) = \frac{1}{2^{x+1}} = \left(\frac{1}{2}\right)^{x+1} \]
A função de probabilidade marginal de X é:
\[ P(X=x) = \left(\frac{1}{2}\right)^{x+1}, \quad \text{para } x=0, 1, 2, \dots \]
Esta é a PMF de uma distribuição \textbf{Geométrica} com parâmetro de sucesso $p=1/2$.
\hfill \qedsymbol

\section{Exercício 10}

\begin{quote}
Seja $F(x,y)$ a função mista de distribuição conjunta de $(X,Y)$. Suponha que $F$ corresponde à densidade Uniforme Contínua sobre o intervalo (0, 1) do eixo y de $\mathbb{R}^2$. \\
a. Obtenha $F$. \\
b. Mostre que $F_X(x)$ é contínua exceto se $x = 0$. \\
c. Prove que $F_Y(y)$ é contínua. \\
d. Você diria que $(X,Y)$ é contínua se, e só se, sua função de distribuição conjunta induz probabilidade zero a cada ponto $(x,y) \in \mathbb{R}^2$?
\end{quote}

\subsection*{Resolução}

\paragraph{Interpretação:} O enunciado descreve uma distribuição onde toda a massa de probabilidade está no segmento de linha $\{ (x,y) : x=0, 0 < y < 1 \}$. Isso significa que a variável aleatória $X$ é discreta e assume o valor 0 com probabilidade 1, i.e., $P(X=0)=1$. A variável $Y$, condicionada a $X=0$, segue uma distribuição Uniforme em $(0,1)$.

\subsubsection*{a. Obtenha $F$}
A função de distribuição conjunta (FC) é $F(x,y) = P(X \le x, Y \le y)$.
\begin{itemize}
    \item Se $x < 0$: O evento $X \le x$ é impossível, pois $X$ é sempre 0. Portanto, $F(x,y)=0$.
    \item Se $x \ge 0$: O evento $X \le x$ é certo ($P(X \le x)=1$). Então, $F(x,y) = P(Y \le y | X=0)$. A FC de uma $U(0,1)$ é $y$ para $y \in [0,1]$. Assim:
    \begin{itemize}
        \item Se $y < 0$, $F(x,y) = 0$.
        \item Se $0 \le y \le 1$, $F(x,y) = y$.
        \item Se $y > 1$, $F(x,y) = 1$.
    \end{itemize}
\end{itemize}
Combinando tudo, a FC conjunta é:
\[ F(x,y) = \begin{cases} 0, & \text{se } x < 0 \text{ ou } y < 0 \\ y, & \text{se } x \ge 0 \text{ e } 0 \le y \le 1 \\ 1, & \text{se } x \ge 0 \text{ e } y > 1 \end{cases} \]

\subsubsection*{b. Mostre que $F_X(x)$ é contínua exceto se $x=0$}
A FC marginal de $X$ é $F_X(x) = \lim_{y\to\infty} F(x,y)$.
\begin{itemize}
    \item Se $x < 0$: $F_X(x) = \lim_{y\to\infty} 0 = 0$.
    \item Se $x \ge 0$: Quando $y\to\infty$, caímos no caso $y>1$, então $F(x,y)=1$.
    \[ F_X(x) = \lim_{y\to\infty} 1 = 1. \]
\end{itemize}
A FC marginal de X é:
\[ F_X(x) = \begin{cases} 0, & \text{se } x < 0 \\ 1, & \text{se } x \ge 0 \end{cases} \]
Esta função é constante (e portanto contínua) para $x < 0$ e $x > 0$. No ponto $x=0$:
\begin{itemize}
    \item $\lim_{x\to 0^-} F_X(x) = \lim_{x\to 0^-} 0 = 0$.
    \item $\lim_{x\to 0^+} F_X(x) = \lim_{x\to 0^+} 1 = 1$.
\end{itemize}
Como os limites laterais são diferentes, $F_X(x)$ é descontínua em $x=0$.

\subsubsection*{c. Prove que $F_Y(y)$ é contínua}
A FC marginal de $Y$ é $F_Y(y) = \lim_{x\to\infty} F(x,y)$.
Quando $x\to\infty$, caímos no caso $x \ge 0$.
\begin{itemize}
    \item Se $y < 0$: $F_Y(y) = \lim_{x\to\infty} 0 = 0$.
    \item Se $0 \le y \le 1$: $F_Y(y) = \lim_{x\to\infty} y = y$.
    \item Se $y > 1$: $F_Y(y) = \lim_{x\to\infty} 1 = 1$.
\end{itemize}
A FC marginal de Y é:
\[ F_Y(y) = \begin{cases} 0, & \text{se } y < 0 \\ y, & \text{se } 0 \le y \le 1 \\ 1, & \text{se } y > 1 \end{cases} \]
Esta é a FC de uma distribuição Uniforme(0,1). A função é contínua em toda a reta real. Nos pontos de junção:
\begin{itemize}
    \item Em $y=0$: $\lim_{y\to 0^-} 0 = 0$ e $\lim_{y\to 0^+} y = 0$. $F_Y(0)=0$. É contínua.
    \item Em $y=1$: $\lim_{y\to 1^-} y = 1$ e $\lim_{y\to 1^+} 1 = 1$. $F_Y(1)=1$. É contínua.
\end{itemize}
Portanto, $F_Y(y)$ é contínua.

\subsubsection*{d. Análise da afirmação}
A afirmação é: " $(X,Y)$ é contínua se, e só se, sua função de distribuição conjunta induz probabilidade zero a cada ponto $(x,y) \in \mathbb{R}^2$ ".

Analisemos as duas implicações da afirmação "se, e só se" (se e somente se):

\paragraph{1. (=>) Se $(X,Y)$ é contínua, então $P(X=x, Y=y)=0$ para todo $(x,y)$.}
Esta afirmação é \textbf{verdadeira}. Por definição, uma variável aleatória vetorial contínua possui uma função de densidade de probabilidade $f(x,y)$, e a probabilidade de qualquer ponto específico é zero.

\paragraph{2. (<=) Se $P(X=x, Y=y)=0$ para todo $(x,y)$, então $(X,Y)$ é contínua.}
Esta afirmação é \textbf{falsa}. Uma variável aleatória vetorial é contínua se, e somente se, sua FC $F(x,y)$ é uma função contínua. A condição $P(X=x, Y=y)=0$ não é suficiente para garantir a continuidade da FC.

O par $(X,Y)$ deste próprio exercício é um contraexemplo perfeito:
\begin{itemize}
    \item A probabilidade de qualquer ponto é zero: $P(X=x, Y=y) = 0$ para todo $(x,y)$. Se $x \ne 0$, a probabilidade é 0. Se $x=0$, a probabilidade é 0 porque $Y$ é uma variável contínua (a probabilidade de $Y$ assumir um valor específico $y$ é nula).
    \item No entanto, a distribuição de $(X,Y)$ \textbf{não é contínua}. Como vimos no item (b), a FC marginal $F_X(x)$ tem uma descontinuidade (um salto) em $x=0$. Isso implica que a FC conjunta $F(x,y)$ também é descontínua ao longo da linha $x=0$.
\end{itemize}

\textbf{Conclusão:} Como a segunda implicação é falsa, a afirmação "se, e só se" é \textbf{falsa}. A condição correta para uma distribuição ser contínua é a continuidade de sua função de distribuição acumulada. \hfill \qedsymbol

\end{document}